\documentclass[12pt]{article}

\title{Tour merging via tree decomposition \\ \vspace{2mm} \small{A hybrid approach between
heuristics and exact solutions for TSP and VRP.}}
\author{Mattias Beimers - 3672565}

\usepackage{amsmath}
\usepackage{amssymb}
\usepackage{enumitem}
\usepackage{graphicx}
\usepackage{algpseudocode}
\usepackage[margin=1.5in]{geometry}

\providecommand{\ee}[1]{\ensuremath{\times 10^{#1}}}

\begin{document}
\maketitle

%
% Abstract
%
\begin{abstract}
    % About TSP and VRP and what this paper does
    This line of text is not very abstract.
\end{abstract}



%
% The introduction
%
\section{Introduction}
\label{sec:introduction}
% Heuristics and Cook-Seymour
For many optimization problems calculating optimal solutions is not feasible in practical
applications, because the computation time grows exponentionally with the problem size. Hence,
heuristics are used to find solutions that are good, but not nescessarily optimal.
To get more certainty that a solution is good, or to improve the solution even more, the heuristics
are applied multiple times and the best solution is selected. Although this works well, Cook and
Seymour noted in their work \cite{cook-seymour} on the Traveling Salesman Problem (TSP) that by
discarding the other solutions, possibly valuable information is lost. Hence the idea emerged to
merge the found approximate solutions and calculate the optimal solution on this reduced problem,
using branch decompositions.

% Our contribution
At the time of publishing the solutions found by Cook and Seymour improved on the best known results
for instances with over 50000 vertices \cite{is this even true???}. Since then the heuristics have
improved massively and outperform the approach by Cook and Seymour \cite{some LKH papers}.
In this paper will try if the strategy for TSP by Cook and Seymour can still improve current heuristics
even more. Apart from that we will try to extend it to work for the Vehicle Routing Problem (VRP) as
well.

% TSP and VRP; desription and defenition
The Traveling Salesman Problem (TSP) is one of the most well studied NP-hard problems, where a
merchant wants to visit a number of cities and get back at his starting point in the shortest
possible amount of time.
We recognize the TSP problem in many practical applications, from planning a school bus route to
scheduling a machine to drill holes in a circuit board.
A generalized version of this problem, where there are not one but a number of merchants or trucks
visiting the cities from the starting point (or depot), is whidely used in the transportation
sector. This problem is known as the Vehicle Routing Problem (VRP).

We define the TSP, given a complete graph $G = (V, E)$, as finding the tour (or Hamiltonian
circuit) with smallest total cost. For this paper we assume the cost $c_e$ of an edge $e = (v, w)$
is the euclidean distance between $v$ and $w$.
Given additionally a demand $d_v$ for each vertex $v$, a maximal capacity $C$ of goods per truck, a
number $M$ of available trucks and a special vertex $v_0$ that is the depot, we can define the VRP
as finding a set of at most $M$ tours with the least total cost. Each tour has to start and end at
the depot and has to have a total demand of at most $C$.
There are many variants with additional constraints or freedoms, but these are out of the scope of
this paper.

% Structure
The paper is organised as follows: in section \ref{sec:heuristics} we will discuss the heuristics
used to generate the initial tours and routes. In section \ref{sec:td} we will discuss how the
solutions are merged and how the treedecomposition is calculated and in section \ref{sec:dp} we will
show the dynamic programming algorithms on the computed decompositions. In section \ref{sec:results}
we will discuss the result and finally we will conclude in section \ref{sec:conclusion}.



%
% Heuristics
%
\section{Heuristics}
\label{sec:heuristics}
Although many different heuristics have been tried to solve the Traveling Salesman Problem, there
are few that can compete with (variants of) the Lin Kernighan heuristic \cite{lin-kernighan}. We
will discuss the Lin Kernighan Helsgaun variant in section \ref{sec:lkh}.
For the Vehicle Routing Problem the currently best heuristics are tabu search algorithms.
Unfortunately they often require to finetune a lot of parameters and are focussed on specific
instances of VRP, rather than giving consistent solutions for all versions \cite{hmmz, where to find
a godo citation for this one}. Other heuristics like the classic savings heuristic or the ???TODO???
heuristic 

\subsection{Lin Kernighan Helsgaun}
\label{sec:lkh}
Todo



%
% Merging and Tree Decomposition calculations
%
\section{Tree decomposition}
\label{sec:td}
Todo: Merging
Todo: Minimum Degree Heuristic



%
% The dynamic programming algorithms
%
\section{Dynamic programming}
\label{sec:dp}
Todo



%
% Results
%
\section{Results}
\label{sec:results}
Todo



%
% Conclusion
%
\section{Conclusion}
\label{sec:conclusion}
Todo



%
% Bibliography
%
\begin{thebibliography}{9}
    % The most important references
    \bibitem{cook-seymour}
        Cook, W., & Seymour, P. (2003).
        \emph{Tour merging via branch-decomposition.}
        INFORMS Journal on Computing, 15(3), 233-248.

\end{thebibliography}

\end{document}
