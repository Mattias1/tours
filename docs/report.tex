\documentclass[12pt]{article}

\title{Tour merging via tree decomposition \\ \vspace{2mm} \small{A hybrid approach between
heuristics and exact solutions for TSP and VRP.}}
\author{Mattias Beimers - 3672565}

\usepackage{amsmath}
\usepackage{amssymb}
\usepackage{enumitem}
\usepackage{graphicx}
\usepackage{algpseudocode}
\usepackage[margin=1.5in]{geometry}

\providecommand{\ee}[1]{\ensuremath{\times 10^{#1}}}

\begin{document}
\maketitle

%
% Abstract
%
\begin{abstract}
    % About TSP and VRP and what this paper does
    This line of text is not very abstract.
\end{abstract}



%
% The introduction
%
\section{Introduction}
\label{sec:introduction}
% TSP and VRP; desription and defenition
One of the most well studied NP-hard problems is the Traveling Salesman Problem (TSP), where a
merchant wants to visit a number of cities and get back at his starting point in the shortest
possible amount of time.
We recognize the TSP problem in many practical applications, from planning a school bus route to
scheduling a machine to drill holes in a circuit board.
A generalized version, where there are not one but a number of merchants or trucks visiting the
cities from the starting point (or depot), is whidely used in the transportation sector. This
problem is known as the Vehicle Routing Problem (VRP).

Given a complete graph $G = (V, E)$, we define the TSP as finding the tour (or Hamiltonian
circuit) with smallest total cost. For this paper we assume the cost $c_e$ of an edge $e = (v, w)$
is the euclidean distance between $v$ and $w$.
Given additionally a demand $d_v$ for each vertex $v$, a maximal capacity $C$ of goods per truck, a
number $M$ of available trucks and a special vertex $v_0$ that is the depot, we can define the VRP
as finding a set of at most $M$ tours with the least total cost. Each tour has to start and end at
the depot and has to have a total demand of at most $C$.
There are many variants with additional constraints or freedoms, but these are out of the scope of
this paper.

% Heuristics and Cook-Seymour
Because the time to compute optimal solutions to these problems grows exponentially with the
problem size, often heuristics are used to find solutions that are usually very good, but not
nescessarily optimal.
To get more certainty that a solution is good, or to improve the solution even more, the heuristics
are applied multiple times, and the best solution is selected. Although this works well, Cook and
Seymour \cite{cook-seymour} noted in their work on TSP that by discarding the other solutions,
possibly valuable information is lost. Hence the idea emerged to merge the found tours and
calculate the optimal solution on this reduced graph, using branch decompositions.

% Our contribution
Todo

% Structure
The paper is organised as follows: More TODOs.



%
% Conclusion
%
\section{Conclusion}
\label{sec:conclusion}
Todo



%
% Bibliography
%
\begin{thebibliography}{9}
    % The most important references
    \bibitem{cook-seymour}
        Cook, W., & Seymour, P. (2003).
        \emph{Tour merging via branch-decomposition.}
        INFORMS Journal on Computing, 15(3), 233-248.

\end{thebibliography}

\end{document}
