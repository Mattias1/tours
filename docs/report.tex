\documentclass[12pt]{article}

\title{Tour merging via tree decomposition \\ \vspace{2mm} \small{A hybrid approach between
heuristics and exact solutions for TSP and VRP.}}
\author{Mattias Beimers - 3672565}

\usepackage{amsmath}
\usepackage{amssymb}
\usepackage{enumitem}
\usepackage{graphicx}
\usepackage{algpseudocode}
\usepackage[margin=1.5in]{geometry}

\providecommand{\ee}[1]{\ensuremath{\times 10^{#1}}}

\begin{document}
\maketitle

% Note:
% Whenever I mention a graph, I should call it a complete graph, that will be sufficiently precise.
% A complete graph is a graph that TODO TODO TODO.
%
% TODO: is a cycle defined as a set of edges, or shoulld I define a tour as a set of edges forming a
%       cycle, rather then just a cycle.
%
% Things to think of:
% - A tour SATISFIES a demand; it doesn't HAVE a demand.
% - Use of the dash (i.e. The Lin-Kernighan-Helsgaun heuristic)
% - Don't say this PAPER, but say this THESIS
% - The word Section with a capital letter
% - Don't use the word treewidth when you mean 'width of the decomposition'.

%
% Abstract
%
\begin{abstract}
    % About TSP and VRP and what this thesis does
    A hybrid approach between heuristics and exact solutions for TSP and VRP using tree
    decompositions.
    % TODO
\end{abstract}



%
% The introduction
%
\section{Introduction}
\label{sec:introduction}
% Heuristics and Cook-Seymour
For many optimization problems calculating optimal solutions is not feasible in practical
applications, because the computation time grows exponentionally with the problem size. Hence,
heuristics are used to find solutions that are good, but not nescessarily optimal.
To get more certainty that a solution is good, or to improve the solution even more, the heuristics
are often applied multiple times and the best solution is selected. Although this works well, Cook
and Seymour noted in their work on the Traveling Salesman Problem (TSP) \cite{cook-seymour} that by
discarding the other solutions, possibly valuable information is lost. Hence the idea emerged to
merge the found approximate solutions and calculate the optimal solution on the merged graph, using
branch decompositions.

% Our contribution
% TODO: is this really at the time of publishing?????
% (1-2 years before - sept 2001), while publish date is 2002 or (2003 according to google scholar)
% TODO: find reference
                                                                                    % TODO TODO TODO TODO TODO TODO TODO TODO TODO
At the time of publishing, the solutions found by Cook and Seymour improved on the best known
results for instances with almost 25000 vertices \cite{the sweden instance}. Since then other
heuristics have improved massively and outperform the approach by Cook and Seymour \cite{lkh,
some other LKH papers}. In this thesis, we will try if the strategy for TSP by Cook and Seymour can
still improve current heuristics even more. Furthermore, we will try to extend it to work for the
Vehicle Routing Problem (VRP).

% TSP and VRP; desription and defenition
The Traveling Salesman Problem (TSP) is one of the most well studied NP-hard problems, where a
merchant wants to visit a number of cities and get back at his starting point in the shortest
possible amount of time.
We recognize the TSP problem in many practical applications, from planning a school bus route to
scheduling a machine to drill holes in a circuit board.
A generalized version of this problem, where there are not one but a number of merchants (or trucks)
visiting the cities from the starting point (or depot), is widely used in the transportation sector.
This problem is known as the Vehicle Routing Problem (VRP).

We define the TSP, given a complete graph $G' = (V, E')$, as finding a tour, or cycle, that visits
all cities exactly once with smallest total cost. For this thesis we assume the cost $c_e$ of an
edge $e = (v, w)$ is the euclidean distance between $v$ and $w$.
Given additionally a demand $d_v$ for each vertex $v$, a maximum capacity $C$ of goods per truck, a
number $M$ of available trucks and a special vertex $v_0$ that is the depot, we can define the VRP
as finding a set of at most $M$ tours with the least total cost. Each tour has to start and end at
the depot and can satisfy a total demand of at most $C$. Each vertex has to be visited by a tour
exactly once.
There are many other variants of the VRP with additional constraints or freedoms, but these are out
of the scope of this thesis.

% Summary of the entire approach.
To solve the TSP and VRP we apply the following stragegy: We start by calculating an initial set of
solutions using heuristics, this gives us a set of promising edges $E \subset E'$. After that we
merge the solutions into a subgraph $G = (V, E)$. On this graph we (hopefully) find a tree
decomposition with small width $k$. With that decomposition we can calculate the optimal solution
in $G$ using a dynamic programming algorithm that has a running time exponential in $k$ but linear
in the number of vertices. This solution often improves on each of the solutions of the heuristic
(TODO: OR NOT, WAIT FOR RESULTS!).

% Structure
The paper is organised as follows: in Section~\ref{sec:heuristics} we will discuss the heuristics
used to generate the initial tours and routes. In Section~\ref{sec:td} we will discuss how the
solutions are merged and how the treedecomposition is calculated and in Section~\ref{sec:dp} we will
show the dynamic programming algorithms on the computed decompositions. In Section~\ref{sec:results}
we will discuss the result and finally we will conclude in Section~\ref{sec:conclusion}.

% TODO: are we (am I) the only one doing something with the cook-seymour approach? (but one recent: yes)
%       (related work)



%
% Heuristics
%
\section{Heuristics}
\label{sec:heuristics}
Although many different heuristics have been tried to solve the Traveling Salesman Problem, there
are few that can compete with (variants of) the Lin-Kernighan heuristic~\cite{lin-kernighan}. We
will discuss the Lin-Kernighan heuristic and the Lin-Kernighan-Helsgaun variant~\cite{lkh} in
Section~\ref{sec:lkh}.
For the Vehicle Routing Problem the currently best heuristics are tabu search algorithms
\cite{TODO: assuming this is even true, the other one}.
Unfortunately they often require to finetune a lot of parameters and are focussed on specific
instances of VRP, rather than giving consistent solutions for all versions \cite{hmmz, where to find
a good citation for this one}. Other heuristics like the classic savings heuristic or the sweep
heuristic do give good solutions for all variants of VRP, but they can't get the results one
gets with the tabu heuristics.
% SWEEP????????

    \subsection{Lin Kernighan}
    \label{sec:lk}
    % Introduction to Lin-Kernighan
    % TODO: check for the use of 'we' and 'they'
    The Lin-Kernighan heuristic is an improvement heuristic. That means that the strategy to solve
    the TSP consists of the following steps:
    \begin{enumerate}
        \item Generate a (pseudo-) random initial tour.
        \item Try to find a modification of the tour to improve it.
        \item If an improved solution is found, replace the tour and repeat from step 2.
        \item If no improved solutions can be found anymore, we are at a local optimum. We can
            either start again from step 1 or stop, depending on some stopping criterium (e.g. the
            solution is good enough, the pool of initial tours is deplenished, or a time limit is
            reached).
    \end{enumerate}
    The interseting part of this strategy is step 2: how do we improve on the current tour. One way
    of doing this is to use a $k$-opt algorithm. In a $k$-opt algorithm we try to find two disjoint
    sets of edges $X$ and $Y$, both containing $k$ edges, such that when we remove the edges in $X$
    from the current tour and replace them the edges from $Y$ the tour will have a lower cost.
    $k$-opt algorithms are well known and often used heuristics because they improve a tour
    effectively while being easy to implement.
    2-opt and 3-opt improvements were proposed for the first time by Croes~\cite{2-opt} and
    Lin~\cite{3-opt}, respecively in 1958 and 1965. $k$-opts with higher values of $k$ have been
    tried as well, for example by TODO~\cite{TODO}, but are less effective.
    However, using $k$-opt for fixed $k$ has its limitations. It is unknown on beforehand which
    value of $k$ will give a good result for the running time involved, as it gets substantially
    harder to find the edge sets and their improvements for increasing values of $k$.

    % The acutal generalized k-opt
    The Lin-Kernighan algorithm is a $k$-opt algorithm, but for a dynamic $k$. The edge sets to be
    replaced are equal to a sequence of 2-opt exchanges, possibly preceeded by a single 3-opt. The
    difference between this approach and repeatedly applying 2-opt optimizations is that not every
    part of the sequence has to improve on the cost of the tour; it is the cost of the entire
    sequence that matters.
    Another difference is that in the 2-opt algorithm we first choose the set $X$ of edges in the
    original tour (i.e. two crossing edges with the euclidean metric) and then find the
    corresponding set $Y$ of new edges to be inserted in the tour. In the Lin-Kernighan algorithm,
    the sets $X$ and $Y$ are built up step by step, as the sequence grows.
    In the algorithm we start by choosing an initial vertex and choose one of the two adjacent edges
    $x_1$ in the original tour. After that we choose an edge $y_1$ from a set of \#? nearest
    neighbours and the accompanying edge $x_2$ (which is uniquely defined most of the time).
    Note that the result after removing the edges in $X$ and adding the edges in $Y$ is not a tour,
    but a path. It can be closed however by adding the edge between the vertex after $x_2$ and the
    vertex before $x_1$.
    We keep adding edges $y_i$ and removing edges $x_{i+1}$, untill either the complete sequence
    including the edge $y_{i+2}$ that will close the tour if added is shorter than the previous tour
    (in which case we succeeded) or that there is no edge $y_i$ to find that improves the tour even
    if the next edges would have weight 0 (in which case we failed).
    If we fail, we can either stop, start again completely, or use backtracking. The latter means
    that we go back a few steps and try another edge $y_i$ (or occasionally $x_{i+1}$).
    TODO - more details?
    % TODO: images
    \dots
    We end by remarking that this is only a short description of the main idea in the Lin-Kernighan
    algorithm. There are many details of importance in the implementation, for whom we refer to the
    original paper~\cite{lin-kernighan}.

    \subsection{Lin Kernighan Helsgaun}
    \label{sec:lkh}
    TODO



%
% Merging and Tree Decomposition calculations
%
\section{Tree decomposition}
\label{sec:td}
% Merging
Once we have generated a set of good tours for our original graph $G'=(V,E')$, we merge these
tours in one graph. All tours $E'_j \subset E'$, for $0 \leq j < \#tours$ as found by the
heuristics are merged together into a new graph $G = (V, E)$, where $E=\bigcup E'_j$. For this graph
$G$ we will compute a tree decomposition so that we can compute the optimal tour in this reduced
problem.
% Roadmap:
Before we show how we compute this decomposition in Section~\ref{sec:td-heuristic}, we first give
the defenition of a tree decomposition.

    % What is a tree decomposition
    \subsection{Tree decomposition and width}
    \label{sec:td-definition}
    A \emph{tree decomposition} of a graph $G=(V, E)$ is a pair $(T=(W, F), X)$, where $T$ is a tree
    and $X=\{X_i \subset V : i \in W\}$ a set of \emph{bags}, satisfying:
    \begin{enumerate}
        \item $\bigcup_{i \in W} X_i = V,$
        \item for all $(u, v) \in E$ there is an $i \in W$ with $u, v \in X_i$ and
        \item for all $v \in V$, the set $W_v = \{i \in W: v \in X_i\}$ forms a connected subtree of $T$.
    \end{enumerate}
    The \emph{width} $k$ of the tree decomposition is $\max_{i \in W} |X_i| - 1$. The \emph{treewidth}
    of a graph $G$, is the minimum width among all tree decompositions of $G$.

    % TODO: example of graph with it's tree decomposition.

    Throughout this thesis we often work with the edge set corresponding to the vertex $i \in W$, rather
    than the vertex set $X_i$ itself. To that end we define $Y_i = \{(u,v) \in E: u,v \in X_i \}$.
    We say that a bag contains a vertex $v$ if $v \in X_i$ and that it contains an edge $e$ if $e \in Y_i$.

    % Minimum Degree Heuristic
    \subsection{Minimum Degree Heuristic}
    \label{sec:td-heuristic}
    Calculating the optimal treewidth or the optimal tree decompositions is an NP Hard problem
    \cite{find ref}, so finding an optimal decomposition in reasonable time is infeasable unlesss
    $P=NP$. We also do not nescessarily need a tree decomposition of optimal width, we just need the
    width to be sufficiently small so that our DP algorithm runs fast enough. Therefore, we compute
    our tree decomposition with a heuristic.

    Bodlaender and Koster~\cite{tw-upper-bounds} evaluated a number of construction heuristics. We
    chose the Minimum Degree Heuristic, originally designed by Markowitz~\cite{min-degree}, which is
    a simple but effective heuristic. It quickly obtains results close to the optimum and is easy to
    implement.
    The algorithm consists of the following steps:
    \begin{enumerate}
        \itemsep 0em
        \item Take the vertex $v \in V$ with minimum degree and add it to $W$.
        \item Create a bag $X_v$ with $v$ and all it's neighbours.
        \item Turn all the neighbours of $v$ into a clique and remove $v$ from $V$.
        \item Add an edge $(v, w)$ to $F$, where $w$ is the neighbour of $v$ with the smallest degree.
        \item Repeat step 1 to 4 untill all vertices are processed and $V = \emptyset$.
    \end{enumerate}
    To complete the tree decomposition we choose the first vertex of $W$ to be the root of the tree.
    Finally we remove the last two vertices from $W$, because their bags only contain 1 or 2 vertices
    and are fully contained in another bag. We can do this because the graph $G$ is obtained by merging
    tours, and therefore every vertex is guaranteed to have at least two neighbours.
    For the VRP we also add the depot vertex to every bag.

    % TODO: the depot vertex causes G to be not 2-connected! - Make sure this doesn't cause any trouble!

    % TODO: trick of that other students paper to not merge all graphs?



%
% The dynamic programming algorithms
%
\section{Dynamic programming}
\label{sec:dp}
Provided the width is small enough, the optimal solution for the merged graph can be computed using
a dynamic programming algorithm on the tree decomposition of the graph. In the following sections
we explain the details of the algorithms.
% TODO: roadmap

    \subsection{Traveling Salesman}
    \label{sec:dp-tsp}
    % 2-connectedness: to make the graph disjoint, you need to remove at least 2 vertices
    % Simle graph: G is unweighted (?); undirected; no double edges; no self loops.
    Let $G=(V, E)$ be a simple graph with edge-weights $c_e$ and $(T=(W, F), X)$ be the tree
    decomposition with width $k - 1$ and $X_i$ and $Y_i$ as defined in Section~\ref{sec:td-definition}.
    Note that because $G$ is the result of a number of merged tours, it is 2-connected. We say that
    a bag $X_j$ is below a bag $X_i$ (in the tree) if $i$ is on the path from $j$ to the root of $T$.
    The main idea of the algorithm is to find a series of disjoint paths and connect them together
    into a Hamiltonian tour of minimum weight. A series of these paths start and end in a bag, and
    visit all vertices in bags below that bag in the tree. Such a series of paths is encoded using
    vertex degrees and a matching. Every vertex can have degree 0, 1 or 2. Vertices with degree 2
    are already \emph{used} in a path, vertices with degree 1 are \emph{endpoints} of a path and
    vertices with degree 0 are \emph{free}, so not yet used in any of the paths. For every pair of
    endpoints we have an edge $\{u, v\}: u,v \in V$ in the \emph{matching} to mark which vertices
    are the endpoints of a path.

    % TODO: F is now both a function AND a set of edges!!!!
    % TODO: what is the degree of an edge set
    We now define the function $F(X_i, D, M)$ to be the minimum total cost of the edges in a series
    of paths starting and ending in bag $X_i$, where $D$ is a set of degrees for the vertices in
    $X_i$ and $M$ a matching. If there is an edge $\{u, v\} \in M$ then there should be a path that
    starts in $u$ and ends in $v$. All vertices in $X_i$ itself should have degrees as given in the
    degrees parameter, and the vertices that occur only in the bags below $X_i$ in the tree should
    all be used (have a degree of 2).

    One way of looking at this is to see the set of degrees $D$ as an instruction to a specific part
    of the tree (the bag $X_i$ and all bags below in the tree) to deliver a set of edges, together
    forming a series of disjoint paths, such that all the degrees of vertices in this bag match with
    the degrees in $D$ and that all vertices that occur only in bags below $X_i$ in the tree are
    used.
    Of course, we do not just want any set of edges, we want the edges that can do it with the
    minimum cost. To get the cost of a tour through the entire graph, we can now call
    $F(X_0, D_0=\{(v, 2),$ for $v \in X_0\}, \emptyset)$. The root of the tree is the special case
    where we allow the paths to form a (single) cycle. Therefore if we give the instruction to the
    root bag $X_0$ to give us a set of edges such that all the vertices inside $X_0$ itself have
    degree 2 (as required by the set $D_0$) and all vertices in bags below the root have degree 2 as
    well (by specification of the function $F$), we actually give the instruction to find the weight
    ofa set of edges that visits all the vertices of $G$ in a single cycle. And because this set of
    edges should have minimum cost, this gives us the cost of the TSP tour through $G$.

    Of course, for a good tree decomposition not all the edges are contained in a single bag. The
    main problem for a non-leaf bag $X_i$ now is not how to find a subset of edges in $Y_i$ that
    satisfy all the requirements for the degrees (and the matching), but how to divide the degrees
    over it's children so that they (recursively) can find the right edge sets that satisfy their
    part of the requirements. Selecting some edges from $Y_i$ is mostly used to stitch the different
    paths from the child bags together so that the complete series of paths meets the requirements
    of $D$ and $M$.
    % TODO: picture

    To find the different ways of dividing the requirements for a bag over it's children, we focus
    solely on the degrees. We also have to decide on the edges in the matching of the children, but
    we will find them as we set the requirements on the degrees.
    We try all possible combinations of dividing the degrees per vertex. For a given vertex $v$,
    assuming we are required to give it a degree of 2, we first try to give it to each of the
    children. So for possibility $p$ and child bag $j$ we try to set the degree of $v$ in $D_{p,j}$
    to 2. Afterwards, assuming we have to give $v$ a degree of at least 1, we try to use it as an
    endpoint coupled with each of the vertices in all of the child bags. So the degrees of two
    vertices, $v$ and some other vertex $u$ in $D_{p,j}$ are set to 1. We try this for all (valid)
    combinations of $u$ and $j$. At this step we also add the edge $\{u, v\}$ to $M_{p,j}$. Finally
    we try not to assign $v$ to any of the child bags, so that we can give its degree with one of
    $Y_i$'s edges.
    Of course, vertices are only given to a child bag if it contains the vertex.

    For the remaining degrees that are not handled by any of the child bags we calculate a subset
    $E_p$ of $Y_i$. For non-leaf bags these edges mainly glue paths from the children together in
    the paths as required by the $D$ and $M$ parameters. For the leaf bags there are of course no child
    bags to delegate the degrees to so it all has to be solved using the bags own edges.
    Note that the edge set is not allowed to introduce cycles, so in particular two endpoints in an
    edge of a matching are not allowed to be connected. This is the reason why we need to give the
    matching as parameter to $F$, without it we do not have sufficient information to determine
    which vertices can and which vertices cannot be connected. The root bag is of course an
    exception, because there all paths are merged in a single cycle.

    In summary, a vertex' degree can be satisfied by passing it on to one (or two) of the child bags
    or in the bag itself by choosing an edge from $Y_i$. Formally this becomes
    \[
        F(X_i, D, M) = \min_{1 \leq p \leq P_i} (
            \sum_{j \in W :\text{ Parent}(j) = i} F(X_j, D_{p,j}, M_{p,j}) + \sum_{e \in E_p} c_e
        )
    \]
    for all $P_i$ ways of dividing $D$ and $M$ into the $D_{p,j}$ and $M_{p,j}$ sets and the
    corresponding $E_p \subset Y_i$.
    If no valid edge set is found, $F(X_i, D, M) = \infty$.

    The overall algorithm then consists of a top down approach where we tabulate all entries for
    the function $F$, starting at the root and then recursively work downwards in the tree. Then the
    value of each table entry is finished bottem up as the recursion returns the values for the
    child entries.

    % Overview:
    % - Idea: Disjoint paths
    % - Vertex degrees: 0, 1, 2
    % - Matchings for the 1's
    % - Talk about subproblems / partial tours / table-entries / functions
    % - The root bag
    % - Top down approach; recursive
    % - It tabulates the degrees+matchings

    \subsection{Vehicle Routing}
    \label{sec:dp-vrp}
    Todo

    % TODO: the graph is not 2-connected in this case, maybe note it, maybe not.

    \subsection{Speed}
    Although DP running time upperbounds of $O(n 3^k 2^{k^2}) - TODO???$ and $O(M n 3^k 2^{\dots})$
    are terrible, in practice these limits are never reached. This is because edges\dots TODO



%
% Results
%
\section{Results}
\label{sec:results}
Todo



%
% Conclusion
%
\section{Conclusion}
\label{sec:conclusion}
Todo



%
% Bibliography
%
\begin{thebibliography}{9}
    % The most important references
    \bibitem{cook-seymour}
        Cook, W., \& Seymour, P. (2003).
        \emph{Tour merging via branch-decomposition.}
        INFORMS Journal on Computing, 15(3), 233-248.

    \bibitem{2-opt}
        Croes, G. A. (1958).
        \emph{A method for solving traveling-salesman problems.}
        Operations research, 6(6), 791-812.

    \bibitem{3-opt}
        Lin, S. (1965).
        \emph{Computer solutions of the traveling salesman problem.}
        Bell System Technical Journal, The, 44(10), 2245-2269.

    \bibitem{lin-kernighan}
        Lin, S., \& Kernighan, B. W. (1973).
        \emph{An effective heuristic algorithm for the traveling-salesman problem.}
        Operations research, 21(2), 498-516.

    \bibitem{lkh}
        Helsgaun, K. (2009).
        \emph{General k-opt submoves for the Lin�Kernighan TSP heuristic.}
        Mathematical Programming Computation, 1(2-3), 119-163.
    % url?

    \bibitem{tw-upper-bounds}
        Bodlaender, H. L., \& Koster, A. M. (2010).
        \emph{Treewidth computations I. Upper bounds.}
        Information and Computation, 208(3), 259-275.

    \bibitem{min-degree}
        Markowitz, H. M. (1957).
        \emph{The elimination form of the inverse and its application to linear programming.}
        Management Science, 3(3), 255-269.

\end{thebibliography}

\end{document}
