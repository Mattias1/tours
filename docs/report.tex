\documentclass[12pt]{article}

\title{Tour merging via tree decomposition \\ \vspace{2mm} \small{A hybrid approach between
heuristics and exact solutions for the TSP and VRP.}}
\author{Mattias Beimers - 3672565}

\usepackage{amsmath}
\usepackage{amssymb}
\usepackage{enumitem}
\usepackage{graphicx}
\usepackage{tikz}
\usepackage{algpseudocode}
\usepackage[margin=1.5in]{geometry}
\usepackage{hyperref}
\usepackage{caption}

\newcommand{\ee}[1]{\ensuremath{\times 10^{#1}}}
\newcommand{\ccP}[0]{\textsc{P}}
\newcommand{\ccNP}[0]{\textsc{NP}}
\newcommand{\cpp}[0]{C\texttt{++}}


\begin{document}
\tikzstyle{vertex2} = [circle,fill=black!25,minimum size=24pt,align=center,font=\small,inner sep=0pt]
\tikzstyle{vertex1} = [circle,fill=black!75,minimum size=4pt,align=center,inner sep=0pt]
\tikzstyle{vertex0} = [circle,fill=black!75,minimum size=1pt,inner sep=0pt]
\tikzstyle{bag} = [circle,fill=black!25,minimum size=45pt,align=center,text width=40pt,font=\small,inner sep=0pt]
\tikzstyle{hugebag} = [circle,fill=black!35,minimum size=99pt,align=center,text width=40pt,font=\small,inner sep=0pt]
\tikzstyle{edge} = [draw=black!75,-]
\tikzstyle{arc} = [draw,->-]
\tikzstyle{weight} = [font=\small]

\maketitle

% Notes:
% Whenever I mention a graph, I should call it a complete graph, that will be sufficiently precise.
% A complete graph is a simple undirected graph in which every pair of distinct vertices is
% connected by a unique edge.
% A simple graph is an unweighted, undirected graph containing no self loops or multiple edges.
%
% Things to think of:
% - A tour SATISFIES a demand; it doesn't HAVE a demand.
% - Use of the dash (i.e. The Lin-Kernighan-Helsgaun heuristic)
% - Don't say this PAPER, but say this THESIS
% - It's until, not untill
% - The word Section with a capital letter
% - Don't use the word treewidth when you mean 'width of the decomposition'.
% - DefInition, not defEnition.
%
% Plot graphs with MathPlotLib: http://matplotlib.org/ (possibly ask Chiel for example).
%

%
% Abstract
%
\begin{abstract}
    % About TSP and VRP and what this thesis does
    The TSP and VRP are well known optimization problems.
    In this thesis we evaluate the use of a dynamic programming algorithm on a tree decomposition of a
    graph, with the goal to improve the solutions to these problems given by several heuristics. The
    used heuristics are the LKH algorithm, the savings algorithm and the sweep algorithm.
\end{abstract}



%
% The introduction
%
\section{Introduction}
\label{sec:introduction}
% Heuristics and Cook-Seymour
For many optimization problems calculating provably optimal solutions is not feasible in practical
applications, because the computation time grows exponentionally with the problem size. Hence,
heuristics are used to find solutions that are good, but not necessarily optimal.
To get more certainty that a solution is good, or to improve the solution even more, the heuristics
are often applied multiple times and the best solution is selected. Although this works well, Cook
and Seymour noted in their work on the Traveling Salesman Problem~\cite{cook-seymour} that by
discarding all but the best solution, possibly valuable information is lost. Hence the idea emerged
to merge all the found solution tours in a single graph and calculate the optimal solution on the
branch-decomposition of that graph.

At the time of writing, the solutions found by Cook and Seymour improved on the best known
results, sometimes even getting optimal solutions, for instances with 15000 and 18000
vertices~\cite{cook-seymour}. Since then other heuristics have improved massively and outperform the
approach by Cook and Seymour~\cite{lkh2}. In this thesis, we will try if the strategy for TSP by
Cook and Seymour can still improve current heuristics even more. Furthermore, we will try to extend
it to work for the Vehicle Routing Problem (VRP).

% TSP and VRP; desription and definition
The Traveling Salesman Problem (TSP) is one of the most well studied NP-hard problems, where a
merchant wants to visit a number of cities and get back at his starting point in the shortest
possible amount of time.
We recognize the TSP problem in many practical applications, from planning a school bus route to
scheduling a machine to drill holes in a circuit board.
% More examples in the LKH 1 paper
A generalized version of this problem, where there are not one but a number of merchants (or trucks)
visiting the cities from the starting point (or depot), is widely used in the transportation sector.
This problem is known as the Vehicle Routing Problem (VRP).

We define the TSP, given a complete graph $G' = (V, E')$, as finding a tour, or cycle, that visits
all cities exactly once with smallest total cost. For this thesis we assume the cost $c_e$ of an
edge $e = (v, w)$ is the euclidean distance between $v$ and $w$.
Given additionally a demand $d_v$ for each vertex $v$, a maximum capacity $C$ of goods per truck, a
number $M$ of available trucks and a special vertex $v_0$ that is the depot, we can define the VRP
as finding a set of at most $M$ tours with the least total cost. Each tour has to start and end at
the depot and can satisfy a total demand of at most $C$. Each vertex has to be visited by a tour
exactly once.
There are many other variants of the VRP with additional constraints or freedoms, but these are out
of the scope of this thesis.

% Summary of the entire approach.
To solve the TSP and VRP we apply the following stragegy: We start by calculating an initial set of
solutions using heuristics. We then merge all the edges of these solutions into a set of promising
edges $E \subset E'$. After that we merge the solutions into a subgraph $G = (V, E)$. On this graph
we (hopefully) find a tree decomposition with small width $k$. With that decomposition we can
calculate the optimal solution in $G$ using a dynamic programming algorithm that has a running time
exponential in $k$ but linear in the number of vertices. This solution often improves on each of the
solutions of the heuristic.
% Our contribution

% Structure
The rest of this thesis is organised as follows: in Section~\ref{sec:heuristics} we will discuss the
heuristics used to generate the initial tours and routes. In Section~\ref{sec:td} we will discuss how
the solutions are merged and how the tree decomposition is calculated and in Section~\ref{sec:dp} we will
show the dynamic programming algorithms on the computed decompositions. In Section~\ref{sec:results}
we will discuss the result and finally we conclude in Section~\ref{sec:conclusion}.



%
% Heuristics
%
\section{Heuristics}
\label{sec:heuristics}
Although many different heuristics have been tried to solve the Traveling Salesman Problem, there
are few that can compete with (variants of) the Lin-Kernighan heuristic~\cite{lkh1, vrp-summary-1},
most notably the implementation of Helsgaun~\cite{lkh2}.
One relatively new family of algorithms are the Stem-and-cycle algorithms~\cite{stem-cycle-1,
stem-cycle-2, tsp-summary}, which perform better than the basic Lin-kernighan
implementations~\cite{tsp-summary}, but cannot beat the advanced implementations like the one from
Helsgaun.
To find initial solutions for the TSP we chose to use the Lin-Kernighan-Helsgaun heuristic
because it is one of the best heuristics available and its source code is available for academic
use~\cite{lkh-url}. We will discuss the original Lin-Kernighan heuristic in Section~\ref{sec:lk} and
the modifications on the original algorithm in Helsgaun's implementation in Section~\ref{sec:lkh}.

For the Vehicle Routing Problem the currently best heuristics are tabu search algorithms
\cite{vrp-summary-1, vrp-summary-2}. Unfortunately they often require to finetune a lot of
parameters and are focussed on specific instances of VRP, rather than giving consistent solutions
for all versions \cite{vrp-summary-1}. Other heuristics like the classic savings heuristic or the
sweep heuristic do give good solutions for all variants of VRP, but they can't get the results one
gets with the tabu heuristics.
For the VRP we chose to use one run of the savings heuristic and multiple runs of the sweep
heuristic, because of their fast running times, reasonable quality of solutions and ease of
implementation. We will discuss these heuristics in Section~\ref{sec:savings} and
Section~\ref{sec:sweep}.

    \subsection{Lin Kernighan}
    \label{sec:lk}
    % Improvement heuristics
    The Lin-Kernighan heuristic~\cite{lin-kernighan} is an improvement heuristic. That means that
    the strategy to solve the TSP consists of the following steps:
    \begin{enumerate}
        \item Generate a (random) initial tour.
        \item Try to find a modification of the tour that improves it.
        \item If an improved solution is found, replace the tour and repeat from step~2.
        \item If no improved solutions can be found anymore, we are at a local optimum. We can
            either start again from step~1 or stop, depending on some stopping criteria (e.g.\ the
            solution is good enough, the pool of initial tours is deplenished, or a time limit is
            reached).
    \end{enumerate}

    % Fixed-k opt
    The interesting part of this strategy is step~2: how do we improve on the current tour. One way
    of doing this is to use a $k$-opt algorithm. In a $k$-opt algorithm we try to find two disjoint
    sets of edges $X$ and $Y$, both containing $k$ edges, such that when we remove the edges in $X$
    from the current tour and replace them with the edges from $Y$ the tour will have a lower cost.
    $k$-opt algorithms are well known and often used heuristics because they improve a tour
    effectively while being easy to implement.
    2-opt and 3-opt improvements were proposed for the first time by Croes~\cite{2-opt} and
    Lin~\cite{3-opt}, respectively in 1958 and 1965. $k$-opts with higher values of $k$ have been
    tried as well, for example by Christofides and Eilon~\cite{2-5-opt}, who tried values for $k$ up
    to 5.
    However, using $k$-opt for fixed $k$ has its limitations. It is unknown beforehand which value
    of $k$ will give a good result for the running time involved, as it costs substantially more
    time to find the edge sets and their improvements for increasing values of $k$.

    % The actual generalized k-opt
    The Lin-Kernighan algorithm is a $k$-opt algorithm, but for a dynamic $k$. The edge sets to be
    replaced are equal to a sequence of 2-opt exchanges, possibly preceded by a single 3-opt. The
    difference between this approach and repeatedly applying 2-opt optimizations is that not every
    part of the sequence has to improve on the cost of the tour; it is the cost of the entire
    sequence that matters.
    Another difference with the fixed $k$-opt algorithms is in how we choose the sets of edges to be
    removed or inserted. In the 2-opt algorithm we first choose the set $X$ of edges in the original
    tour to be removed (i.e.\ two crossing edges with the euclidean metric) and then find the
    corresponding set $Y$ of new edges to be inserted in the tour. In the Lin-Kernighan algorithm,
    the sets $X$ and $Y$ are built up step by step, as the sequence grows.

    In the algorithm we start by choosing an initial vertex $t_1$ and choose one of the two
    adjacent edges $x_1 = (t_1, t_2)$ in the original tour.
    After we have chosen the first edge $x_1$, we repeatedly choose edges $y_i = (t_{2i}, t_{2i+1})$
    and $x_{i+1} = (t_{2i+1}, t_{2i+2})$, according to some criteria.
    Note that $X$ contains one edge more than $Y$. Because of that, if we remove the edges in $X$
    from the current tour and add the edges in $Y$ to it, the result will not be a tour but a path.
    However, if $X$ and $Y$ are constructed in the right way, we can close the path by adding the
    edge $y_{i+1} = (t_{2i+2}, t_1)$ to get a tour.
    We choose an edge $y_i$ from a set of edges to the 5 nearest neighbours of $t_{2i}$. This edge
    $y_i$ may not be in the original tour already and the current sequence must have a positive
    gain, i.e.\ $\sum_{j=0}^i x_j - y_j > 0$. Furthermore, $y_i$ should be chosen such that a next
    edge $x_{i+1}$ exists (i.e.\ $x_{i+1}$ is not chosen already).
    The edge $x_{i+1}$ is uniquely defined for all $i \geq 1$, as there are only two edges adjacent
    to $t_{2i}$ in a tour, and if the wrong one is chosen the tour cannot be closed anymore (see
    Figure~\ref{fig:unique-x}).
    An exception is made for $x_2$, where the wrong edge choice is allowed as it can be fixed by
    the choice for $y_3$ and $x_4$.
    If there are multiple valid choices for $y_i$ or $x_{i+1}$, we initially choose the one with the
    highest gain. For $y_i$ we look ahead and choose the edge where $x_{i+1} - y_i$ is largest.
    For $x_{i+1}$ we choose the edge with the largest weight. Other choices are ignored at first,
    but may be examined via backtracking.
    We keep adding edges $y_i$ and $x_{i+1}$ to the sets, until either the complete sequence
    (including the last edge $y_{i+1}$ that will close the tour) is shorter than the previous tour
    (in which case we succeeded) or that there is no edge $y_i$ to find that improves the tour even
    if the closing edge $y_{i+1}$ would have weight 0 (in which case we failed).
    If we fail, we can either stop, start again from scratch (with a different vertex for $t_1$), or
    use backtracking. The latter means that we go back a few steps and try another edge $y_i$ (or
    occasionally another edge $x_{i+1}$).
    As backtracking is quite time consuming, we only allow it at the first two levels ($i \leq 2$).
    An example of the dynamic $k$-opt procedure is shown in Figure~\ref{fig:lk-example}.

    This completes the overview of the Lin-Kernighan heuristic. We will describe two more
    optimizations below, one to speed up the algorithm, the other to improve the solution. There are
    many more details which are essential for the efficiency of the algorithm however, as it is not
    a simple algorithm to implement~\cite{lkh1}. We omit them here as they are not essential to this
    thesis. For the exact details we refer to the original paper~\cite{lin-kernighan}.

    The first optimization reduces the choices for the edges. In the first few solutions found by
    the algorithm there always are a lot of edges that appear in every solution. These common edges
    are recorded and then they are no longer allowed to be broken for the other solutions. This
    means they cannot be used as choice for $x_{i+1}$ any more, which in turn limits the choices for
    $y_i$ and speeds up the overall algorithm significantly. Note that in order to not bias the
    solution too much (we do not want to find the exact same solution again after all), we only use
    this restriction for $i \geq 4$.

    In the second optimization of the algorithm we apply the double bridge move.
    The dynamic $k$-opt procedure that we described here only allows us to find so called sequential
    moves: a series of connected edges, alternating inside and outside the original tour. Not all
    possible $k$-opts consist of such a sequence however, and not considering these moves can
    sometimes be the difference between a good solution and the optimal solution. The easiest
    example of a non sequential move is a 4-opt known as the the double-bridge move (see
    Figure~\ref{fig:double-bridge}).
    It is tried as a post-optimization after we finished with the entire algorithm and only for
    edges that are not amoungst the common edges from the previous optimization. Lin and Kernighan
    found that in some cases this improved the result significantly, while in other cases it did
    nothing. It doesn't hurt to try though, as it is a relatively cheap optimization.

    % Nice summary in LKH 2 paper: 1-5 on page 5 of 45 (p. 123)
    % 1: The sequential exchange cirterion
    % 2: The feasibility criterion
    % 3: The positive gain criterion
    % 4: The disjunctivity criterion
    % 5: The candidate set criterion

    \begin{figure}
    \centering
    \begin{tabular}{c c}
        \begin{tikzpicture}[auto,swap]
        \node[vertex1] (0) at (1.2, -2.9) {};
        \node[vertex1] (1) at (1.4, 0.1) {};
        \node[vertex1] (2) at (2.0, 0.3) {};
        \node[vertex0] (3) at (4.1, -1.2) {};
        \node[vertex0] (4) at (3.9, -2.4) {};
        \node[vertex1] (5) at (1.7, -3.2) {};
        \path[edge,bend left=48] (0) to (1);
        \path[edge] (0) to node[below] {$x_1$} (5);
        \path[edge] (0) to node[right] {$y_1$} (2);
        \path[edge] (1) to node[above] {$x_2$} (2);
        \path[edge,dashed] (1) to node[left] {$y_2$} (5);
        \path[edge,bend left=36] (2) to (3);
        \path[edge,bend left=19] (3) to (4);
        \path[edge,bend left=32] (4) to (5);
        \end{tikzpicture}
        &
        \begin{tikzpicture}[auto,swap]
        \node[vertex1] (0) at (1.2, -2.9) {};
        \node[vertex1] (1) at (1.4, 0.1) {};
        \node[vertex1] (2) at (2.0, 0.3) {};
        \node[vertex1] (3) at (4.2, -1.2) {};
        \node[vertex1] (4) at (4.2, -1.8) {};
        \node[vertex1] (5) at (1.7, -3.2) {};
        \path[edge,bend left=48] (0) to (1);
        \path[edge] (0) to node[below] {$x_1$} (5);
        \path[edge] (0) to node[right] {$y_1$} (2);
        \path[edge] (1) to node[above] {$x_2$} (2);
        \path[edge] (1) to node[left] {$y_2$} (4);
        \path[edge,bend left=36] (2) to (3);
        \path[edge] (3) to node[right] {$x_3$} (4);
        \path[edge,dashed] (3) to node[left] {$y_3$} (5);
        \path[edge,bend left=42] (4) to (5);
        \end{tikzpicture}
        \\
    \end{tabular}
    \caption{An example of a 3-opt move, as constructed by the Lin-Kernighan heuristic. Note that
    vertices are displayed in a circle in the order they appear in the original tour.}
    \label{fig:lk-example}
    \end{figure}

    \begin{figure}
    \centering
    \begin{tikzpicture}[auto,swap]
    \node[vertex1] (0) at (1.2, -2.9) {};
    \node[vertex1] (1) at (1.4, 0.1) {};
    \node[vertex1] (2) at (2.0, 0.3) {};
    \node[vertex1] (3) at (4.2, -1.8) {};
    \node[vertex1] (4) at (4.0, -2.4) {};
    \node[vertex1] (5) at (1.7, -3.2) {};
    \path[edge,bend left=48] (0) to (1);
    \path[edge] (0) to node[below] {$x_1$} (5);
    \path[edge] (0) to node[right] {$y_1$} (2);
    \path[edge] (1) to node[above] {$x_2$} (2);
    \path[edge] (1) to node[left] {$y_2$} (3);
    \path[edge,bend left=43] (2) to (3);
    \path[edge] (3) to node[right] {$x_3$} (4);
    \path[edge,dashed] (4) to node[left] {$y_3$} (5);
    \path[edge,bend left=32] (4) to (5);
    \end{tikzpicture}
    \caption{An example where the wrong choice for $x_3$ is made, resulting in two disjunct tours.}
    \label{fig:unique-x}
    \end{figure}

    \begin{figure}
    \centering
    \begin{tikzpicture}[auto,swap]
    \node[vertex1] (0) at (7.0, -0.94) {};
    \node[vertex1] (1) at (7.7, -0.94) {};
    \node[vertex1] (2) at (7.0, -4.5) {};
    \node[vertex1] (3) at (7.7, -4.5) {};
    \node[vertex1] (4) at (9.1, -2.4) {};
    \node[vertex1] (5) at (9.1, -3.1) {};
    \node[vertex1] (6) at (5.5, -2.4) {};
    \node[vertex1] (7) at (5.5, -3.1) {};
    \path[edge] (0) to node[left] {$y_1$} (2);
    \path[edge,bend right=30] (0) to (6);
    \path[edge] (0) to node[above] {$x_2$} (1);
    \path[edge,bend left=30] (1) to (4);
    \path[edge] (1) to node[right] {$y_2$} (3);
    \path[edge] (2) to node[below] {$x_1$} (3);
    \path[edge,bend left=30] (2) to (7);
    \path[edge,bend right=30] (3) to (5);
    \path[edge] (4) to node[right] {$x_4$} (5);
    \path[edge] (4) to node[above] {$y_3$} (6);
    \path[edge] (5) to node[below] {$y_4$} (7);
    \path[edge] (6) to node[left] {$x_3$} (7);
    \end{tikzpicture}
    \caption{The double bridge move}
    \label{fig:double-bridge}
    \end{figure}


    \subsection{Lin Kernighan Helsgaun}
    \label{sec:lkh}
    The Lin-Kernighan-Helsgaun algorithm~\cite{lkh1, lkh2} is based on the algorithm of Lin and
    Kernighan, but improves it on several points.

    % Minimum 1 tree instead of 5 nearest neighbours (section 4.1 (and 4.2 for x_1) in LKH 1 paper)
    We define a 1-tree for a graph $G=(V, E)$ as a spanning tree on the vertex set $V \setminus \{1\}$
    combined with two edges from $E$ incident to the 1-vertex. An example of a 1-tree is shown in
    Figure~\ref{fig:1-tree}. Note that a 1-tree is not a tree (as it contains a cycle) and that the
    choice of the 1-vertex is arbitrary. A minimum 1-tree is a 1-tree of minimum length. The
    \emph{$\alpha$-nearness} of an edge $e$ is defined as the difference between the length of the
    smallest 1-tree containing $e$ and the length of the minimum 1-tree. Or in other words:
    $\alpha$-nearness is the increase in length of the minimum 1-tree if it is required to contain
    the edge $e$.
    A straightforward computation of all $\alpha$-nearness values of the edges takes $O(n^3)$ time,
    but in his paper Helsgaun shows how to compute it in $O(n^2)$ time~\cite{lkh}.

    The first major difference with the Lin-Kernighan algorithm is the candidate set for the edges
    in $Y$. In the original algorithm the choices for an edge $y_i$ are limited to the edges to the
    first 5 nearest neighbours. The choice for this candidate set assumes that the shorter an edge
    is, the higher the chance is that it occurs in a tour. This is reasonable, but it is not always
    a good estimation. Helsgaun notices that the $\alpha$-nearness, is a better estimation and uses
    it instead of the nearest neighbours as the candidate set.
    The candidate set then is further modified to always include edges that are in both of the two
    previous best solutions. These edges are tried first.
    The candidate set for the first edge to be chosen, $x_1$, is also changed so that no edges from
    the previous best tour are removed in the first level of the improvement step.

    \begin{figure}
    \centering
    \begin{tikzpicture}[auto,swap]
    \node[vertex1] (0) at (5.1, -0.89) {};
    \node[vertex1] (1) at (4.1, -1.8) {};
    \node[vertex1] (2) at (2.9, -2.8) {};
    \node[vertex1] (3) at (4.9, -2.8) {};
    \node[vertex1] (4) at (4.9, -4.2) {};
    \node[vertex1] (5) at (6.1, -2.0) {};
    \node[vertex1] (6) at (7.1, -2.9) {};
    \node[vertex1] (7) at (8.2, -2.0) {};
    \node[vertex1] (8) at (8.3, -3.6) {};
    \node[vertex1] (9) at (7.1, -4.4) {};
    \node[vertex1,label=0:\;$\leftarrow$ the 1-vertex] (10) at (5.9, -5.3) {};
    \path[edge] (0) to (1);
    \path[edge] (0) to (5);
    \path[edge] (1) to (2);
    \path[edge] (1) to (3);
    \path[edge] (3) to (4);
    \path[edge] (4) to (10);
    \path[edge] (5) to (6);
    \path[edge] (6) to (7);
    \path[edge] (6) to (8);
    \path[edge] (6) to (9);
    \path[edge] (9) to (10);
    \end{tikzpicture}
    \caption{A 1-tree}
    \label{fig:1-tree}
    \end{figure}

    The second major change is the choice of the basic move. In the original Lin-Kernighan algorithm
    every move was composed as a sequence of 2-opt moves (and possibly a 3-opt).
    Helsgaun~\cite{lkh1} modified it to use moves that consist of 5-opt moves (unless a $k$-opt move
    for smaller $k$ results in an improvement already). Later~\cite{lkh2} he modified it again to
    use general $k$-opt moves up to a certain $k$ that we can choose ourselves as the basic step.
    The steps to do this are rather involved and contain a large case-analysis and won't be
    described here in detail. For the details we refer to the paper of Helsgaun~\cite{lkh2}.
    The main idea of his approach however is that we allow edges $x_i$ that initially break up the
    tour (like in Figure~\ref{fig:unique-x}). We then look ahead to the following edges $y_{i+1}$
    and $x_{i+1}$ to make sure that acceptable edges exist that can fix the tour. These edges in
    their turn do not have to be chosen; we can choose edges with higher gain that break up the tour
    again, providing we look ahead and find acceptable edges that can fix the tour. This allows,
    amoung others, the double bridge move from Figure~\ref{fig:double-bridge} to be included
    natively in the search.

    % - Basic move no longer 2- (or 3-) opt move, but a 5-opt move (but stopped as soon as it
    %   improves, so it can also do just a 2-opt move for example. Also backtracking removed.
    %   5-opt as basic step adds quite a lot, as observed by \cite{2-5-opt}.
    %   Also something with non-sequential moves (no longer just post-optimization).
    % - General k-opt moves

    % Initial tours
    The third change is not aimed at improving the quality of the solutions, but aims to speed up
    the algorithm.
    As in most improvement heuristics, the Lin-Kernighan algorithm is tried several times on
    different initial tours. These tours where constructed randomly because Lin and Kernighan
    considered construction heuristics to be unnecessary. Helsgaun notes that even though a good
    construction heuristic does not significantly improve the quality of the final tour, it does
    improve the running time of the algorithm.
    The heuristic used by Helsgaun tries to construct a tour greedily by choosing edges that are in
    the minimum 1-tree and that are in the previous best tour. If no such edge is found either a
    random other candidate edge is chosen or, if they are not valid either, an edge to any free
    vertex.

    % Other points from LKH 2 paper:
    Amoungst some more additions added by Helsgaun is the option of merging tours.
    Similar to what we do in this thesis he merges the edges of a few solutions in a single graph
    and on this merged graph he solves the TSP again.
    Unlike what we do in this thesis he doesn't solve the merged problem to optimality, but applies
    the general $k$-opt submoves again. This time he uses larger value of $k$, which he can do
    because the graph is sparse. We disable this option in our experiments, as we perform this
    calculation ourselves.
    There are some more additions, not specifically related to the Lin-Kernighan algorithm, but more
    general applicable to the TSP, for which we refer to Helsgaun's paper~\cite{lkh2}.


    \subsection{Savings}
    \label{sec:savings}
    The savings heuristic~\cite{savings} is a simple but relatively effective construction heuristic
    for the VRP\@. A construction heuristic is a heuristic that it creates a set of tours without
    relying on anything except the data on the vertices and truck capacity. There are two versions
    of the savings algorithm, the parallel version and the sequential version. We implement the
    parallel version here, as the resulting tours are usually better than the tours generated by the
    sequential savings algorithm~\cite{vrp-summary-1, vrp-summary-2}.
    Our savings algorithm consists of the following steps:
    \begin{enumerate}
        \itemsep 0em
        \item
            Initially we create $n - 1$ tours, one for each non-depot vertex. Each tour starts at
            the depot, goes to the associated vertex and then goes back to the depot.
        \item
            For every pair of vertices $i, j$ we then calculate the saving $S_{i, j} =
            c_{(0, i)} + c_{(0, j)} - c_{(i, j)}$. This represents the gain in cost we get by
            removing the two edges $(0, i)$ and $(0, j)$, and adding the edge $(i, j)$.
        \item
        \label{enum:savings-merge}
            We now take the saving $S_{i, j}$ with the highest gain and check if we can apply it.
            With applying $S_{i, j}$ we mean that we try to find two tours, one of which ends with
            the edge $(0, i)$ and one of which begins with the edge $(0, j)$ (or the other way
            around). We then connect the two tours together by removing these edges and adding the
            edge $(i, j)$.
            This is of course only possible if there are two tours that begin or end with these
            edges and if the demand of both tours together does not exceed the capacity of a truck.
        \item
            We continue to repeat step~\ref{enum:savings-merge} until no more savings can be
            applied. If the total amount of tours is less or equal than the allowed amount of
            trucks, we are done. If it exceeds the limit, we failed.
        \item
            If the heuristic has successfully found a set of tours, we optimize each tour separately
            using the Lin-Kernighan-Helsgaun heuristic as described in Section~\ref{sec:lkh}.
    \end{enumerate}

    We decided not to make an exception for the savings that have a negative gain (i.e.\ savings
    that make the total cost worse). This means that we apply them just like we apply the positive
    savings.
    The reasoning behind this is that it's better to have a slightly worse solution than an invalid
    solution. Applying the negative savings means we use one less truck, which is a hard constraint
    on the problem and thus more important.
    Note that this situation occurs very little (if ever) because the negative savings are tried
    last, so this is not likely to have a big impact on the solution of the tours.

    The characteristic behaviour for the savings algorithm is to put a lot of effort into getting a
    set of tours with low cost. This produces good tours, which is nice, but it also has a negative
    side effect. This disadvantage is that it does not pay much effort into making sure the capacity
    of a truck is used optimally. This means that sometimes it generates solutions that need more
    trucks than it is allowed to use, which makes the given set of tours an invalid solution. A
    possible workaround would be to allow solutions that use more tours than strictly allowed in the
    hope that the edges would be of use in creating the final tour after the merging step from
    Section~\ref{sec:td}. We decided against this because that would be cheating - after all, the
    goal is not to finetune heuristics for the merging process but the other way around.
    Another disadvantage of this heuristic is that it is deterministic, and can therefore
    only produce a single solution. To get more solutions we use the sweep heuristic in
    Section~\ref{sec:sweep}.


    \subsection{Sweep}
    \label{sec:sweep}
    The sweep heuristic is another heuristic for the VRP\@.  It is most commonly attributed to
    Gillett and Miller in 1974~\cite{sweep}, but can be found earlier in a book from
    Wren~\cite{sweep-orig-1} or a paper from Wren and Holliday~\cite{sweep-orig-2} in 1971 and 1972
    respectively.
    It depends on the geometry of the vertices and is only applicable for the euclidean VRP, not in
    general. This is not a problem for us because we aim to solve the euclidean VRP anyway. Unlike
    the savings heuristic, the sweep heuristic is an non-deterministic algorithm in the sense that
    it depends on the starting vertex (or starting angle) and can produce different sets of tours
    for the same instance.
    Note that a starting angle and a starting vertex correspond one on one with each other. If you
    begin with a starting angle the starting vertex is the first vertex that it will hit when
    rotating clockwise. If a starting vertex is given, all angles between the angle through the
    starting vertex and the angle through the first vertex counter clockwise of the starting vertex
    create the same set of tours.

    The sweep heuristic is a two-phase algorithm. That means that it splits up the task of creating
    a set of tours by first clustering the vertices into groups that can be served by a single
    truck, and then deciding for each cluster of vertices separately in which order the truck should
    visit the vertices. The first stage of clustering the vertices is what the sweep algorithm is
    designed to solve. The second stage is essentially a set of TSP problems, which can be solved
    using a dedicated algorithm. We solve this second step using the Lin-Kernighan-Helsgaun
    heuristic from Section~\ref{sec:lkh}.

    Because of its geometric nature, the sweep heuristic can be best explained by picturing a
    drawing of the vertices on a plane. The algorithm then consists of the following steps:
    \begin{enumerate}
        \itemsep 0em
        \item
            We start by drawing a ray from the depot through the starting vertex. We add these two
            vertices to the first tour; the tour we are currently building.
        \item
        \label{enum:sweep-add}
            We then rotate the line clockwise until it hits a vertex. If the vertex can still be
            added to the previous tour (i.e.\ the total demand of the vertices in the tour does not
            exceed the capacity of a truck), we add it to the tour.
        \item
            We repeat step~\ref{enum:sweep-add} until adding the next vertex will exceed the
            capacity limit (or until there are no vertices left, in which case we are finished and
            go to step~\ref{enum:sweep-end}).
        \item
            Once the capacity limit is reached, we add the next vertex and the depot to a new tour,
            and we repeat from step~\ref{enum:sweep-add} until there are no vertices left.
        \item
        \label{enum:sweep-end}
            Once we are finished with the sweep part, we solve each tour separately using LKH.
    \end{enumerate}

    The tours created by this heuristic are not necessarily bad, but they are very clustered. This
    means that it will never explore a large part of the solution space. To make for a bit more of a
    variety of edges, we use the savings heuristic as described in Section~\ref{sec:savings}.

    One thing left to discuss is the choice of the vertex to start with. For the choice of the
    starting vertex, given the start vertex for the previous set of tours, there are basically two
    options.
    One option is to start with the first vertex that comes next with respect to clockwise
    rotation, this would create a sequence of tours that look a lot like each other. The other
    choice would be to spread out the start vertices evenly, causing the set of tours to be move
    diverse but less related.
    We chose the second option to choose the starting vertices uniformly. One reason is that the
    first option is basically a disguised version of the String Relocation heuristic, which is an
    improvement heuristic that removes a sequence of vertices from one tour and places it in the
    neighbouring tour. The second option however should add a diverse set of edges to the final
    tour, which should allow for broader viewpoint.



%
% Merging and Tree Decomposition calculations
%
\section{Tree decomposition}
\label{sec:td}
% Merging
Once we have generated a set of good tours for our original graph $G'=(V,E')$, we merge these
tours in one graph. All tours $E'_j \subset E'$, for $0 \leq j < \#tours$ as found by the
heuristics are merged together into a new graph $G = (V, E)$, where $E=\bigcup E'_j$. For this graph
$G$ we will compute a tree decomposition so that we can compute the optimal tour in this reduced
problem.
% Roadmap:
Before we show how we compute this decomposition in Section~\ref{sec:td-heuristic}, we first give
the definition of a tree decomposition in Section!\ref{sec:td-definition}.

    % What is a tree decomposition
    \subsection{Tree decomposition and width}
    \label{sec:td-definition}
    A \emph{tree decomposition} of a graph $G=(V, E)$ is a pair $(T=(W, H), X)$, where $T$ is a tree
    with an arbitrary root vertex and $X=\{X_i \subset V : i \in W\}$ a set of \emph{bags},
    satisfying:
    \begin{enumerate}
        \item $\bigcup_{i \in W} X_i = V,$
        \item for all $(u, v) \in E$ there is an $i \in W$ with $u, v \in X_i$ and
        \item for all $v \in V$, the set $W_v = \{i \in W: v \in X_i\}$ forms a connected subtree of $T$.
    \end{enumerate}
    The \emph{width} $k$ of the tree decomposition is $\max_{i \in W} |X_i| - 1$. The \emph{treewidth}
    of a graph $G$, is the minimum width among all tree decompositions of $G$.
    An example of a tree decomposition is shown in Figure~\ref{fig:td}.

    \begin{figure}
    \centering
    \begin{tikzpicture}[auto,swap]
    \node[vertex2] (0) at (0.59, -3.7) {0};
    \node[vertex2] (1) at (2.3, -0.78) {1};
    \node[vertex2] (2) at (3.0, -2.5) {2};
    \node[vertex2] (3) at (5.2, -2.4) {3};
    \node[vertex2] (4) at (3.0, -4.2) {4};
    \node[vertex2] (5) at (4.7, -5.6) {5};
    \node[vertex2] (6) at (1.6, -6.0) {6};
    \path[edge] (0) to (1);
    \path[edge] (0) to (4);
    \path[edge] (0) to (6);
    \path[edge] (1) to (2);
    \path[edge] (1) to (3);
    \path[edge] (2) to (3);
    \path[edge] (2) to (4);
    \path[edge] (3) to (4);
    \path[edge] (3) to (5);
    \path[edge] (4) to (5);
    \path[edge] (5) to (6);
    \node[bag] (0) at (9.5, -1.2) {0: 0, 1, 3, 4};
    \node[bag] (1) at (8.0, -3.3) {1: 1, 2, 3, 4};
    \node[bag] (2) at (1.1e+01, -3.3) {2: 0, 3, 4, 5};
    \node[bag] (3) at (1.1e+01, -5.8) {3: 0, 5, 6};
    \path[edge] (0) to (1);
    \path[edge] (0) to (2);
    \path[edge] (2) to (3);
    \end{tikzpicture}
    \caption{An example of a graph and its tree decomposition.}
    \label{fig:td}
    \end{figure}

    Throughout this thesis we often work with the edge set corresponding to the vertex $i \in W$, rather
    than the vertex set $X_i$ itself. To that end we define $Y_i = \{(u,v) \in E: u,v \in X_i \}$.
    We say that a bag contains a vertex $v$ if $v \in X_i$ and that it contains an edge $e$ if $e \in Y_i$.


    % Minimum Degree Heuristic
    \subsection{Minimum Degree Heuristic}
    \label{sec:td-heuristic}
    Calculating the optimal treewidth or the optimal tree decomposition for unknown treewidth is an
    \ccNP-Hard problem \cite{tw-hardness}, so finding an optimal decomposition in reasonable time is
    infeasable unless \ccP=\ccNP\@. We do not necessarily need a tree decomposition of optimal
    width, we just need the width to be sufficiently small so that our dynamic programming algorithm
    runs fast enough. Therefore, we compute our tree decomposition with a heuristic.

    Bodlaender and Koster~\cite{tw-upper-bounds} evaluated a number of construction heuristics. We
    chose to use the Minimum Degree Heuristic, originally designed by Markowitz~\cite{min-degree},
    because it is a simple but effective heuristic. It is fast, obtains results close to the optimum
    and is easy to implement.
    A non-recursive version of the algorithm consists of the following steps:
    \begin{enumerate}
        \itemsep 0em
        \item
            Initially let $(T=(W, H), X)$ with $W$, $H$ and $X$ set to $\emptyset$.
        \item
        \label{enum:min-degree-begin}
            Take the vertex $v \in V$ with minimum degree and add it to $W$; i.e.\ add a new
            vertex to $W$ with the same name as $v$.
            The reason to give it the same name is that it allows us to add edges ahead of time in
            step~\ref{enum:min-degree-add-edge}.
        \item
        \label{enum:min-degree-add-to-bag}
            Create a bag $X_v$ with $v$ and all its neighbours in $G$.
        \item
        \label{enum:min-degree-add-edge}
            Add an edge $(v, w)$ to $H$, where $w$ is the neighbour of $v$ in $G$ with the smallest
            degree (so the first neighbour to be processed). Note that $w$ is not yet added to $W$,
            but will be added in the future.
        \item
        \label{enum:min-degree-end}
            Modify $G$ by turning all the neighbours of $v$ into a clique and removing $v$ from
            $V$ (and its incident edges from $E$).
        \item
            Repeat step~\ref{enum:min-degree-begin} to~\ref{enum:min-degree-end} until all vertices
            are processed (i.e.\ $V = \emptyset$).
    \end{enumerate}
    To complete the tree decomposition we choose the first vertex of $W$ to be the root of the tree,
    and, when we are solving the VRP, we also add the depot vertex to every bag.

    As a small optimization step we remove any bag that is fully contained in it's parent. This
    happends quite often in this algorithm, because in step~\ref{enum:min-degree-add-to-bag} we add
    all the neighbours of a vertex $v$ to a bag, and then connect all these neighbours to each other
    in step~\ref{enum:min-degree-end}. Once one of these next vertices gets processed while it
    doesn't have any neighbours that $v$ did not have, it will add all the neighbours of $v$ again,
    but this time without $v$ itself. Hence the duplication.



%
% The dynamic programming algorithms
%
\section{Dynamic programming}
\label{sec:dp}
Provided the width is small enough, the optimal solution for the TSP or VRP on the merged graph can
be computed using a dynamic programming algorithm on the tree decomposition of the graph. In the
following sections we explain the details of the algorithms.

    \subsection{Traveling Salesman}
    \label{sec:dp-tsp}
    % 2-connectedness: to make the graph disjoint, you need to remove at least 2 vertices
    % Simle graph: G is unweighted (?); undirected; no double edges; no self loops.
    Let $G=(V, E)$ be a simple graph with edge-weights $c_e$ and $(T=(W, H), X)$ be the tree
    decomposition with width $k - 1$ and $X_i$ and $Y_i$ be the bags with respecitvely vertices or
    edges as defined in Section~\ref{sec:td-definition}. We say that a bag $X_j$ is below a bag
    $X_i$ (in the tree) if $i$ is on the path from $j$ to the root of $T$.
    Note that because $G$ is the result of a number of merged tours, it is 2-connected and all
    vertices have a degree of at least two.
    The main idea of the algorithm is to find a series of disjoint paths and connect them together
    into a Hamiltonian tour of minimum weight. A series of these paths start and end in a bag, and
    visit all vertices in bags below that bag in the tree. Such a series of paths is encoded using
    vertex degrees and a matching.  Every vertex can have degree 0, 1 or 2. Vertices with degree 2
    are already \emph{used} in a path, vertices with degree 1 are \emph{endpoints} of a path and
    vertices with degree 0 are \emph{free}, so not yet used in any of the paths. For every pair of
    endpoints we have an edge $\{u, v\}: u,v \in V$ in the \emph{matching} to mark which vertices
    are the endpoints of a path.

    We now want to use the function $F(X_i, D_i, M_i)$ as the minimum total cost of the edges in a
    series of paths starting and ending in bag $X_i$, where $D_i$ is a set of degrees for the
    vertices in $X_i$ and $M_i$ a matching on the vertices of degree 1. If there is an edge
    $\{u, v\} \in M_i$ then there should be a path that starts in $u$ and ends in $v$. All vertices
    in $X_i$ itself should have degrees as given in the degrees parameter, and the vertices that
    occur only in the bags below $X_i$ in the tree should all be used (have a degree of 2) in one of
    the paths.

    One way of looking at this is to see the set of degrees $D_i$ as an instruction to a specific
    part of the tree (the bag $X_i$ and all bags below in the tree) to deliver a set of edges,
    together forming a series of disjoint paths, such that all the degrees of vertices in this bag
    match with the degrees in $D$ and that all vertices that occur only in bags below $X_i$ in the
    tree are used.
    Of course, we do not just want any set of edges, we want the edges that can do it with the
    minimum cost. To get the cost of a tour through the entire graph, we can now call
    $F(X_0, D_0=\{(v, 2),$ for $v \in X_0\}, \emptyset)$. The root of the tree is the special case
    where we allow the paths to form a (single) cycle. Therefore if we give the instruction to the
    root bag $X_0$ to give us a set of edges such that all the vertices inside $X_0$ itself have
    degree 2 (as required by the set $D_0$) and all vertices in bags below the root have degree 2 as
    well (by specification of the function $F$), we actually give the instruction to find the weight
    of a set of edges that visits all the vertices of $G$ in a single cycle. And because this set of
    edges should have minimum cost, this gives us the cost of the TSP tour through $G$.

    Of course, for a good tree decomposition not all the edges are contained in a single bag. The
    main problem for a non-leaf bag $X_i$ now is not how to find a subset of edges in $Y_i$ that
    satisfy all the requirements for the degrees (and the matching), but how to divide the degrees
    % TODO: COMMENT (in the middle of "how to" and "divide"): Compute this from that values of F for
    % it's child bags. This means that we want to ...
    over it's children so that they (recursively) can find the right edge sets that satisfy their
    part of the requirements. Selecting some edges from $Y_i$ is mostly used to stitch the different
    paths from the child bags together so that the complete series of paths meets the requirements
    of $D$ and $M$.
    An example of a path through a part of a tree decomposition is depicted in
    Figure~\ref{fig:dp-tsp-example}.

    \begin{figure}
    \centering
    \begin{tikzpicture}[auto,swap]
    \node[hugebag] (b-0) at (7.0, -3.1) {};
    \node[hugebag] (b-1) at (9.6, -6.4) {};
    \node[hugebag] (b-2) at (4.2, -6.7) {};
    \node[vertex0] (b-3) at (7.0, -0.79) {};
    \node[vertex0] (b-4) at (9.6, -8.7) {};
    \path[edge] (b-0) to (b-2);
    \path[edge] (b-0) to (b-1);
    \path[edge] (b-0) to (b-3);
    \path[edge] (b-1) to (b-4);

    \node[vertex2] (0) at (6.29, -2.34) {0};
    \node[vertex2] (1) at (7.81, -2.34) {1};
    \node[vertex2] (2) at (7.64, -3.91) {2};
    \node[vertex2] (3) at (6.53, -4.19) {3};
    \node[vertex2] (4) at (3.96, -5.81) {0};
    \node[vertex2] (5) at (5.33, -6.84) {3};
    \node[vertex2] (6) at (4.04, -7.75) {6};
    \node[vertex2] (7) at (3.11, -6.68) {7};
    \node[vertex2] (8) at (8.83, -6.54) {2};
    \node[vertex2] (9) at (10.26, -5.59) {1};
    \node[vertex2] (10) at (10.60, -6.86) {10};
    \node[vertex0] (11) at (6.29, -0.99) {};
    \node[vertex0] (12) at (7.81, -0.99) {};
    \node[vertex0] (13) at (8.83, -8.44) {};
    \node[vertex0] (14) at (10.60, -8.44) {};
    \node[vertex2] (15) at (5.81, -3.38) {7};
    \path[edge,dashed] (0) to (4);
    \path[edge,dashed] (1) to (9);
    \path[edge,dashed] (2) to (8);
    \path[edge,dashed] (3) to (5);
    \path[edge] (2) to (3);
    \path[edge] (4) to (6);
    \path[edge] (5) to (6);
    \path[edge] (9) to (10);
    \path[edge,dashed] (0) to (11);
    \path[edge,dashed] (1) to (12);
    \path[edge,dashed] (8) to (13);
    \path[edge,dashed] (10) to (14);

    \node[text width=50] at (2,0.5){$
         F(X_{top}, \; \{(v_0,1), (v_1,1), (v_2,2), (v_3,2), (v_7,0)\}, \; \{\{v_0, v_1\}\})$,
        $F(X_{left}, \; \{(v_0,1), (v_3,1), (v_6,2), (v_7,0)\}, \; \{\{v_0, v_3\}\})$,
        $F(X_{right}, \; \{(v_2,1), (v_1,1), (v_{10},2)\}, \; \{\{v_2, v_1\}\})$
    };

    \end{tikzpicture}
    \caption{An example of a path through a part of the tree decomposition, together with the
    instructions per bag.}
    \label{fig:dp-tsp-example}
    \end{figure}

    To find the different ways of dividing the requirements for a bag over it's children, we have to
    go through these three steps:
    \begin{enumerate}
        \item Find the degrees
        \item Find the matchings
        \item Find the edge sets
    \end{enumerate}
    Step~1 and~2 are done simultaniously. We first focus on the degrees and at the same time have
    to decide on the edges in the matching of the children. We will find them while we set the
    requirements on the degrees.
    We try all possible combinations of dividing the degrees per vertex. For a given vertex $v$,
    assuming we are required to give it a degree of 2, we first try to give it to each of the
    children. So for possibility $p$ and child bag $j$ we try to set the degree of $v$ in $D_{p,j}$
    to 2. Afterwards, assuming we have to give $v$ a degree of at least 1, we try to use it as an
    endpoint coupled with each of the vertices in all of the child bags. So the degrees of two
    vertices, $v$ and some other vertex $u$ in $D_{p,j}$ are set to 1. We try this for all (valid)
    combinations of $u$ and $j$. At this step we also add the edge $\{u, v\}$ to $M_{p,j}$. Finally
    we try not to assign $v$ to any of the child bags, so that we can give its degree with one of
    $Y_i$'s edges.
    Of course, vertices are only given to a child bag if it contains the vertex.

    For the remaining degrees that are not handled by any of the child bags we calculate a subset
    $E_p$ of $Y_i$. This is the third step. For non-leaf bags these edges mainly glue paths from the
    children together in the paths as required by the $D_i$ and $M_i$ parameters. For the leaf bags
    there are of course no child bags to delegate the degrees to so it all has to be solved using
    the bags own edges.
    Note that the edge set is not allowed to introduce cycles, so in particular two endpoints in an
    edge of a matching are not allowed to be connected. This is the reason why we need to give the
    matching as parameter to $F$, without it we do not have sufficient information to determine
    which vertices can and which vertices cannot be connected. The root bag is of course an
    exception, because there all paths are merged in a single cycle.

    In summary, a vertex' degree can be satisfied by passing it on to one (or two) of the child bags
    or in the bag itself by choosing an edge from $Y_i$. Formally this becomes
    \[
        F(X_i, D_i, M_i) = \min_{1 \leq p \leq P_i} (
            \sum_{j \in W :\text{\ Parent}(j) = i} F(X_j, D_{p,j}, M_{p,j}) + \sum_{e \in E_p} c_e
        )
    \]
    for all $P_i$ ways of dividing $D_i$ and $M_i$ into the $D_{p,j}$ and $M_{p,j}$ sets and the
    corresponding $E_p \subset Y_i$. The edges are not allowed to form a cycle.
    If no valid edge set is found, $F(X_i, D_i, M_i) = \infty$.

    The overall algorithm then consists of a top down approach where we tabulate all entries for
    the function $F$, starting at the root and then recursively work downwards in the tree. Then the
    value of each table entry is finished bottem up as the recursion returns the values for the
    child entries.

    % Overview:
    % - Some definitions
    % - Idea: Disjoint paths
    %   - Vertex degrees: 0, 1, 2
    %   - Matchings for the 1's
    % - Define F
    % - F looks like Instructions
    % - The root bag
    % - Pass on the instructions
    % - How to divide the requirements amoung child bags
    % - Remaining vertices: find edge set inside Yi
    % - Formal definition of F
    % - Top down approach; recursive + it tabulates the degrees+matchings


    \subsection{Vehicle Routing}
    \label{sec:dp-vrp}
    Let $G=(V, E)$ again be a simple graph with edge-weights $c_e$ and $(T=(W, H), X)$ be the tree
    decomposition with width $k - 1$ and $X_i$ and $Y_i$ be the bags with respectively vertices or
    edges as defined in Section~\ref{sec:td-definition}.
    The definition of the instruction-function $F(X_i, D_i, M_i, W_i)$ remains the same except that
    the edges $e \in M_i$ now have an edge weight $w_e \in W_i$ as well.
    Furthermore let $M$ be the number of trucks that we have to use (and therefore te number of
    tours that we should find) and let $C$ be the capacity of each truck. We denote the demand of
    all vertices $v_i \in V$ by $d_i$, where the demand $d_0$ of the depot vertex $v_0$ is equal to 0.
    Note that $G$, the resulting graph after merging the solutions of the heuristics, is not
    2-connected as was the case for the TSP\@. This is because removing the depot vertex causes it
    to be disconnected. In our algorithm we still assume that all vertices have a degree of two or
    more. This is not always true, as there can be a tour from the depot to a single vertex $v$ and
    then back, but this case is trivial, as the cost of that tour is $2 \cdot c_{(0, v)}$, and it
    can be excluded from the input to the function $F$.
    With the demand of a path, we mean the total demand of all the vertices on this path.
    When the matching edge of a path is in the instruction function $F$ for the current bag $X_i$,
    we call this path a \emph{main path}.
    Whenever two vertices in a main path are connected with a matching edge, rather than a normal
    edge from $Y_i$, this edge of the matching encodes the start and end vertices of a
    \emph{subpath} through a child bag of $X_i$.
    For example, the bag $X_{top}$ in Figure~\ref{fig:dp-tsp-example} has one main path and two
    subpaths.

    The main idea of the algorithm is similar to that of the algorithm for the TSP\@. We try to find
    a series of paths that together form not one, but a set of $M$ tours. The instruction of how
    these paths should be delivered by a bag is again done by specifying the degrees of the vertices
    and a matching. The degrees of all non-depot vertices can be between 0 and 2 like before, but
    the degree of the depot vertex can be any value between 0 and $2M$. Furthermore we add the
    restriction that the depot can only appear as an endpoint of a path. This restriction makes the
    final merging of the paths easier and effectively ensures that the depot is part of every tour.
    The greatest difference however lies in the fact that we need to restrict the total demand of
    the vertices in each of the tours to be at most $C$, the capacity of a truck.
    To do this we need to be able to enforce this restriction on each of the paths separately in the
    parameters of the function $F$.
    As an edge in the matching relates one on one to a path, it comes natural to extended the edge
    with an edge-weight $w_{\{u, v\}}$, which denotes that the demand of path from $u$ to $v$ can be
    at most $w_{\{u, v\}}$.
    To get the solution for a VRP instance we call $F(X_0, D_0, M_0, W_0)$, where $D_0$ contains the
    maximum degree for all vertices, and $M_0$ contains $M$ times the edge $\{0, 0\}$ with weight
    $w_{\{0, 0\}}$ is equal to the capacity $C$ of a truck.

    The difficulty lies in how to divide the demand from the main paths in the current bag $X_i$,
    over the demands for the subpaths through the child bags. Note here that the focus has shifted
    from the bags to the individual paths through the bags. For the VRP we have the three steps from
    the TSP, but with an additional fourth step.
    \begin{enumerate}
        \item Find the degrees
        \item Find the matchings
        \item Find the edge sets
        \item Find the demands per path
    \end{enumerate}
    In the dynamic programming algorithm for the TSP the main difficulty lay at dividing the degrees
    over the bags, but we didn't care which vertex was included in which path. We only had to ensure
    that there was some path satisfying the degree of a vertex as was instructed; for as long as its
    endpoints coincided with the matching of course.
    Likewise, for the VRP algorithm, we first try to divide the degrees over the bags while initially
    ignoring the individual paths. As before we find the edges in the matchings as a side effect of
    finding the degrees, but at this point we don't know the weight of the edges yet. Once we
    divided both the degrees and found the matchings, we try, like before, to find an edge selection
    to support these choices of $D_i$ and $M_i$.
    Only after we are done with the first three steps we try to find the weight of the matching. If a
    main path only has one subpath, this is quite simple. The demand of the subpath is the demand of
    the main path minus the demand of the vertices already visited inside the current bag.
    Each main path can have multiple subpaths however, and these subpaths are independend of the
    bags. Multiple subpaths from the same main path can be spread over multiple bags (but don't have
    to), and there can be subpaths from different main paths in the same bag.
    Because we don't know how much capacity is needed for the subpaths yet, we simply have to try
    all possibilities. In the example of Figure~\ref{fig:dp-tsp-example} the bag $X_{top}$ might
    be allowed to have a demand of at most 6 (if all vertices have a demand of 1 and the child bag
    of $X_{right}$ has no additional vertices not in $X_{right}$ itself this is just enough). We now
    need to try to give one bag a demand of 6 and the other 0 (and the other way around), then try
    to give one 5 and the other 1, then 4 and 2 and finally we need to try giving both bags a demand
    of 3.

    Another effect of the capacity constraint is the choice of the edge sets $E_p \subset Y_i$,
    which are constructed after the degrees and matching are chosen. For the TSP algorithm we just
    have to find a path that uses all vertices that we instructed it to use, but it doesn't matter
    whether a vertex is used in one path or another. For the VRP placing a vertex in one path rather
    than in another might mean that the demand of that path is more than the capacity of a truck,
    leaving the solution invalid. As a consequence we do not have to evaluate just the cheepest edge
    selection, but all of them.

    % Overview:
    % - Some definitions
    %   - Mostly the same as for the VRP, but with some additions.
    %   - Define main paths and subpaths (and independendness).
    % - Idea: same as TSP, but with demands (matching-edge weight)
    %                      and a depot (2M degree + endpoint of path only)
    % - Focus on paths rather than bags
    %   - Recap from TSP: first divide degrees, find matchings as side effect, then edges follow,
    %   - After that we divide demands
    %   - Each path can have multiple subpaths, subpaths are independend of bags: multiple subpaths
    %     from the same main path can be spread over multiple bags, there can be subpaths from
    %     different main paths in the same bag.
    %   - Try all different combinations of the main-path-demand on all subpaths
    % - The edge sets
    % - Overview:
    %   - Root bag



%
% Results
%
\section{Results}
\label{sec:results}
We implemented the algorithms described in this thesis in \cpp\;using the GNU GCC compiler on both
Linux and Windows (using Cygwin in the latter case). We used both the source code and the binary for
the Lin-Kernighan-Helsgaun algorithm, which is available for academic use on the website of
Helsgaun~\cite{lkh-url}.
All experiments ran on a laptop with an Intel i5-4210U CPU, with a clock rate of 1.70GHz.
% Dutch: "Test opstelling en resultaten" - dus welke grafen getest, waarvandaan, waarom die grafen?

    \subsection{Traveling Salesman Problem}
    In our test we considered some of the TSPLIB instances provided by the 8th DIMACS Implementation
    challange~\cite{tsp-instances}. Specifically we tested the ones that have between 1000 and 2500
    vertices.  The LKH algorithm was limited to 100 trials and 10 runs each. This means that every
    tour was generated by evaluating 100 different starting vertices $t_0$ (see also
    Section~\ref{sec:lk}), and that we ran the algorithm 10 times.

    The results of our experiments with the LKH heuristic are displayed in
    Table~\ref{table:tsp-heuristics}. The name of the tour represents, apart from the TSPLIB name,
    also the number of vertices in the input graph.  In the width column we give the width of the
    decomposition on the merged graph. The \#tours column lists the number of LKH runs that added
    edges that where not used in any of the previous runs. The best and average columns represent
    the best and average values found by the LKH tours. The time displays the time it took for the
    heuristic to run. Note that the minimum degree heuristic is always completed in a few
    milliseconds, and does not influence the results.
    Note that in two cases our program had a problem reading the input graph and aborted with an
    exception.

    \begin{table}
    \begin{tabular}{l l l l l l}
        name    & width & \#tours   & best      & avg.      & time \\ \hline
        d1291   & 7     & 7         & 50801     & 50884     & 45s \\
        d1655   & 5     & 8         & 62128     & 62130     & 48s \\
        d2103   & 8     & 10        & 80460     & 80493     & 1m 58s \\
        dsj1000 & 4     & 3         & 18660188  & 18667688  & 1m 31s \\
        fl1400  & 14    & 10        & 20164     & 20167     & 50m 19s \\
        fl1577  & 6     & 10        & 22263     & 22266     & 20m 24s \\
        nrw1379 & -     & -         & -         & -         & - \\
        pcb1173 & 5     & 4         & 56892     & 56898     & 10s \\
        pr1002  & 7     & 5         & 259045    & 259152    & 10s \\
        pr2392  & 3     & 8         & 378032    & 378032    & 20s \\
        rl1304  & 7     & 5         & 252948    & 253227    & 29s \\
        rl1323  & 10    & 5         & 270199    & 270552    & 26s \\
        rl1889  & 10    & 8         & 316549    & 316747    & 1m 20s \\
        si1032  & -     & -         & -         & -         & - \\
        u1060   & 8     & 10        & 224094    & 224136    & 1m 33s \\
        u1432   & 16    & 10        & 152970    & 152970    & 47s \\
        u1817   & 12    & 10        & 57259     & 57277     & 59s \\
        u2152   & 11    & 10        & 64253     & 64308     & 1m \\
        u2318   & 66    & 10        & 234256    & 234256    & 3m 21s \\
        vm1084  & 8     & 5         & 239303    & 239407    & 36s \\
        vm1748  & 12    & 8         & 336739    & 336758    & 35s \\
    \end{tabular}
    \caption{The results of the LKH algorithm}
    \label{table:tsp-heuristics}
    \end{table}

    The results of the experiments with the dynamic program algorithm are displayed in
    Table~\ref{table:tsp-dp}. The name, width and \#tours column are the same as before. The cost
    column displays the cost of the optimal tour through the merged graph. The time displays the
    time it took for the dynamic programming algorithm to complete.
    Note that in the cases where the width of the decomposition was larger than 10, we did not
    continue with the dynamic programming algorithm, as it would exceed the available running time.
    Another thing to note is that even though our program has double precision for the coordinates
    of the vertices, it truncates the cost of an edge to an integer value. There are different
    conventions about what type should be used for the computations~\cite{vrp-summary-1}. In this
    case the LKH algorithm uses double precision for the edge costs. This causes the differences
    between the cost found by LKH and the cost as calculated by the dynamic programming to be
    larger than it actually is. The new tours found by the dynamic programming do improve the LKH
    tours in most cases, but the difference is somewhat less as that it seems.

    \begin{table}
    \begin{tabular}{l l l l l}
        name    & width & \#tours   & cost      & time \\ \hline
        d1291   & 7     & 7         & 50658     & 6s \\
        d1655   & 5     & 8         & 61748     & 0s \\
        d2103   & 8     & 10        & 79310     & 2m 39s \\
        dsj1000 & 4     & 3         & 18659188  & 0s \\
        fl1400  & 14    & 10        & -         & - \\
        fl1577  & 6     & 10        & 21298     & 0s \\
        nrw1379 & -     & -         & -         & - \\
        pcb1173 & 5     & 4         & 56699     & 0s \\
        pr1002  & 7     & 5         & 258829    & 0s \\
        pr2392  & 3     & 8         & 377553    & 0s \\
        rl1304  & 7     & 5         & 252774    & 1s \\
        rl1323  & 10    & 5         & 270012    & 30m 18s \\
        rl1889  & 10    & 8         & 316240    & 1m 25s \\
        si1032  & -     & -         & -         & - \\
        u1060   & 8     & 10        & 223581    & 26s \\
        u1432   & 16    & 10        & -         & - \\
        u1817   & 12    & 10        & -         & - \\
        u2152   & 11    & 10        & -         & - \\
        u2318   & 66    & 10        & -         & - \\
        vm1084  & 8     & 5         & 239019    & 27s \\
        vm1748  & 12    & 8         & -         & - \\
    \end{tabular}
    \caption{The results of the dynaming algorithm}
    \label{table:tsp-dp}
    \end{table}

    Interesting to see is that there is no correlation between the running times of the heuristic
    and the running times of the dynamic programming algorithm. The exponential behaviour of the
    decomposition width is clearly visible however. Usually the time to run LKH dominates the time
    to run the dynamic programming algorithm, until the width of the decomposition reaches 10.


    \subsection{Vehicle Routing Problem}
    To test how the tour merging approach works for the VRP, we downloaded several instances
    for the Capacitated VRP from the website of the NEO research group~\cite{vrp-instances}.
    The instances vary from having 32 vertices to 80 vertices in total. The number of trucks vary
    from 5 to 10 trucks. The capacity of a truck is equal for all the instances and is set to 100.
    Even though these where amoung the smallest instances available, even the smallest ones were
    aborted due to too long running times.

    Because of the limited timeframe and computing power we decided to modify the instances, in
    search of what our algorithm can handle.
    First we chose to limit the number of vertices (and accordingly the number of trucks
    necessary for the demand of the cities). We did this by simply removing the vertices after the
    $n$-th vertex. This showed us that our algorithm was able to handle instances with a limit of
    approximately 13 vertices.
    Apart from that we decided to change the capacity parameter from 100 to 10 for some of the
    instances, and round down the demands of the vertices accordingly. This allowed our program to
    handle a few more vertices, but not very much. The cases where we changed the capacity are
    labelled with a (c) behind the name.

    The results of the computations concerning the heuristics are displayed in
    Table~\ref{table:vrp-heuristics}. In the name column we display the name of the instance, which
    includes the original number of vertices and the number of trucks used. In the next column we
    display the number of vertices that we limited the computation to. The width column shows the
    width of the decomposition again. The \#savings and \#sweep columns show the number of runs in
    which the algorithms added new edges to the set of tours. The best and average columns show the
    best and average costs of set of tours found by the heuristics. Noteworthy is that in all the
    cases where the savings algorithm found an acceptable tour it had the best result.
    Running times are not displayed as all heuristics, including the minimum degree heuristic used
    to compute the tree decomposition, ran in less than a second.

    \begin{table}
    \begin{tabular}{l l l l l l l}
        name        & n     & width & \#savings & \#sweep   & best  & avg. \\ \hline
        A-n32-k5    & 32    & 6     & 1         & 9         & 832   & 973 \\
        A-n32-k5    & 19    & 6     & 0         & 6         & 535   & 656 \\
        A-n32-k5    & 12    & 4     & 0         & 8         & 460   & 478 \\
        A-n33-k5    & 33    & 5     & 0         & 9         & 715   & 757 \\
        A-n33-k5    & 13    & 4     & 0         & 6         & 413   & 435 \\
        A-n33-k6    & 14    & 5     & 1         & 4         & 330   & 450 \\
        A-n32-k5(c) & 13    & 4     & 0         & 8         & 444   & 474 \\
        A-n32-k5(c) & 14    & 5     & 0         & 8         & 448   & 485 \\
        A-n32-k5(c) & 15    & 5     & 0         & 7         & 448   & 500 \\
    \end{tabular}
    \caption{The results of the savings and sweep algorithms}
    \label{table:vrp-heuristics}
    \end{table}

    The results of the computations by the dynamic programming algorithm for the VRP are displayed
    in Table~\ref{table:vrp-dp}. The name, amount of vertices, width, best and average columns are
    the same as before. The \#tours column contains the combined number of merged sets of tours by
    the savings algorithm and the sweep algorithm. The cost column contains the cost of the tour as
    calculated by the dynamic programming algorithm. The time column contains the running time of
    the dynamic programming algorithm.
    Overall the cost of the tours given by heuristics is improved, but at a huge cost in complexity
    of the code and running time.
    Note that all of the larger instances where aborted after a given number of hours.

    \begin{table}
    \begin{tabular}{l l l l l l l l}
        name        & n     & width & \#tours   & best  & avg. & cost   & time \\ \hline
        A-n32-k5    & 32    & 6     & 10        & 832   & 973  & -      & 8h (aborted) \\
        A-n32-k5    & 19    & 6     & 6         & 535   & 757  & -      & 4h (aborted) \\
        A-n32-k5    & 12    & 4     & 8         & 460   & 478  & 406    & 2m 26s \\
        A-n33-k5    & 33    & 5     & 9         & 715   & 656  & -      & 8h (aborted) \\
        A-n33-k5    & 13    & 4     & 6         & 413   & 435  & 406    & 19m 4s \\
        A-n33-k6    & 14    & 5     & 5         & 330   & 450  & -      & 8h (aborted) \\
        A-n32-k5(c) & 13    & 4     & 8         & 444   & 474  & 436    & 2s \\
        A-n32-k5(c) & 14    & 5     & 8         & 448   & 485  & 441    & 6s \\
        A-n32-k5(c) & 15    & 5     & 7         & 448   & 500  & 441    & 19m 56s \\
    \end{tabular}
    \caption{The results of the dynamic programming algorithm}
    \label{table:vrp-dp}
    \end{table}



%
% Conclusion
%
\section{Conclusion}
\label{sec:conclusion}
In this thesis we have looked at the tour merging technique introduced by Cook and
Seymour~\cite{cook-seymour}, for the Traveling Salesman and Vehicle Routing problems.
We have seen that merging a set tours created by heuristics improves the overall solutions. In the
case of the TSP this is already known, as it was shown by Cook and Seymour before. In the case of the
VRP this is new, but it is not very helpful. There are better heuristics available than the ones we
used, but these are often rejected because of the complexity of the code and running times (and also
because they are not generally applicable). Our algorithm is both very complex in code as slow in
running time and not applicable for large instances, and thus is not an effective alternative for
more complicated heuristics.



%
% Bibliography
%
\begin{thebibliography}{9}
    % The most important references
    \bibitem{cook-seymour}
        Cook, W., \& Seymour, P. (2003).
        \emph{Tour merging via branch-decomposition.}
        INFORMS Journal on Computing, 15(3), 233-248.

    % TSP heuristics
    \bibitem{tsp-summary}
        Rego, C., Gamboa, D., Glover, F., \& Osterman, C. (2011).
        \emph{Traveling salesman problem heuristics: leading methods, implementations and latest advances.}
        European Journal of Operational Research, 211(3), 427-441.

    \bibitem{stem-cycle-1}
        Pesch E., Glover, F. (1997).
        \emph{TSP ejection chains.}
        Discrete Applied Mathematics, 76(1), 165-181.

    \bibitem{stem-cycle-2}
        Rego, C. (1998).
        \emph{Relaxed tours and path ejections for the traveling salesman problem.}
        European Journal of Operational Research, 106(2), 522-538.

    \bibitem{2-opt}
        Croes, G. A. (1958).
        \emph{A method for solving traveling-salesman problems.}
        Operations research, 6(6), 791-812.

    \bibitem{3-opt}
        Lin, S. (1965).
        \emph{Computer solutions of the traveling salesman problem.}
        Bell System Technical Journal, The, 44(10), 2245-2269.

    \bibitem{2-5-opt}
        Christofides, N., \& Eilon, S. (1972).
        \emph{Algorithms for large-scale travelling salesman problems.}
        Operational Research Quarterly, 511-518.

    \bibitem{lin-kernighan}
        Lin, S., \& Kernighan, B. W. (1973).
        \emph{An effective heuristic algorithm for the traveling-salesman problem.}
        Operations research, 21(2), 498-516.

    \bibitem{lkh1}
        Helsgaun, K. (2000).
        \emph{An effective implementation of the Lin-Kernighan traveling salesman heuristic.}
        European Journal of Operational Research, 126(1), 106-130.

    \bibitem{lkh2}
        Helsgaun, K. (2009).
        \emph{General k-opt submoves for the Lin-Kernighan TSP heuristic.}
        Mathematical Programming Computation, 1(2-3), 119-163.

    \bibitem{lkh-url}
        Helsgaun, K.
        \emph{LKH}
        \url{http://www.akira.ruc.dk/~keld/research/LKH/}

    % VRP heuristics
    \bibitem{vrp-summary-1}
        Laporte, G., Gendreau, M., Potvin, J. Y., \& Semet, F. (2000).
        \emph{Classical and modern heuristics for the vehicle routing problem.}
        International transactions in operational research, 7(4-5), 285-300.

    \bibitem{vrp-summary-2}
        Cordeau, J. F., Gendreau, M., Laporte, G., Potvin, J. Y., \& Semet, F. (2002).
        \emph{A guide to vehicle routing heuristics.}
        Journal of the Operational Research society, 512-522.

    \bibitem{savings}
        Clarke, G. U., \& Wright, J. W. (1964).
        \emph{Scheduling of vehicles from a central depot to a number of delivery points.}
        Operations research, 12(4), 568-581.

    \bibitem{sweep-orig-1}
        Wren, A., (1971).
        \emph{Computers in Transport Planning and Operation.}
        Ian Allan, London.

    \bibitem{sweep-orig-2}
        Wren, A., Holliday, A., (1972).
        \emph{Computer scheduling of vehicles form one or more depots to a number of delivery points.}
        Operational Research Quarterly 23, 333-344.

    \bibitem{sweep}
        Gillett, B. E., \& Miller, L. R. (1974).
        \emph{A heuristic algorithm for the vehicle-dispatch problem.}
        Operations research, 22(2), 340-349.

    % Treewidth heuristics
    \bibitem{tw-upper-bounds}
        Bodlaender, H. L., \& Koster, A. M. (2010).
        \emph{Treewidth computations I. Upper bounds.}
        Information and Computation, 208(3), 259-275.

    \bibitem{min-degree}
        Markowitz, H. M. (1957).
        \emph{The elimination form of the inverse and its application to linear programming.}
        Management Science, 3(3), 255-269.

    \bibitem{tw-hardness}
        Arnborg, S., Corneil, D. G., \& Proskurowski, A. (1987).
        \emph{Complexity of finding embeddings in ak-tree.}
        SIAM Journal on Algebraic Discrete Methods, 8(2), 277-284.

    % Instances
    \bibitem{tsp-instances}
        Johnson, D., McGeoch, L., Glover, F. \& Rego, C.
        \emph{8th DIMACS Implementation Challenge: The Traveling Salesman Problem}
        \url{http://dimacs.rutgers.edu/Challenges/TSP/download.html}

    \bibitem{vrp-instances}
        NEO research group
        \emph{Vehicle Routing Problem}
        \url{http://neo.lcc.uma.es/vrp/solution-methods/}


\end{thebibliography}

\end{document}
