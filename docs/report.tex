\documentclass[12pt]{article}

\title{Tour merging via tree decomposition \\ \vspace{2mm} \small{A hybrid approach between
heuristics and exact solutions for TSP and VRP.}}
\author{Mattias Beimers - 3672565}

\usepackage{amsmath}
\usepackage{amssymb}
\usepackage{enumitem}
\usepackage{graphicx}
\usepackage{algpseudocode}
\usepackage[margin=1.5in]{geometry}

\providecommand{\ee}[1]{\ensuremath{\times 10^{#1}}}

\begin{document}
\maketitle

% Note:
% Whenever I mention a graph, I should call it a complete graph, that will be sufficiently precise.
% A complete graph is a graph that TODO TODO TODO.

%
% Abstract
%
\begin{abstract}
    % About TSP and VRP and what this paper does
    A hybrid approach between heuristics and exact solutions for TSP and VRP using tree
    decompositions.
\end{abstract}



%
% The introduction
%
\section{Introduction}
\label{sec:introduction}
% Heuristics and Cook-Seymour
For many optimization problems calculating optimal solutions is not feasible in practical
applications, because the computation time grows exponentionally with the problem size. Hence,
heuristics are used to find solutions that are good, but not nescessarily optimal.
To get more certainty that a solution is good, or to improve the solution even more, the heuristics
are applied multiple times and the best solution is selected. Although this works well, Cook and
Seymour noted in their work on the Traveling Salesman Problem (TSP) \cite{cook-seymour} that by
discarding the other solutions, possibly valuable information is lost. Hence the idea emerged to
merge the found approximate solutions and calculate the optimal solution on this reduced problem,
using branch decompositions.

% Our contribution
At the time of publishing the solutions found by Cook and Seymour improved on the best known results
for instances with almost 25000 vertices \cite{the sweden instance}. Since then the heuristics have
improved massively and outperform the approach by Cook and Seymour \cite{lkh, some other LKH papers}.
In this paper will try if the strategy for TSP by Cook and Seymour can still improve current heuristics
even more. Furthermore, we will try to extend it to work for the Vehicle Routing Problem (VRP).

% TSP and VRP; desription and defenition
The Traveling Salesman Problem (TSP) is one of the most well studied NP-hard problems, where a
merchant wants to visit a number of cities and get back at his starting point in the shortest
possible amount of time.
We recognize the TSP problem in many practical applications, from planning a school bus route to
scheduling a machine to drill holes in a circuit board.
A generalized version of this problem, where there are not one but a number of merchants or trucks
visiting the cities from the starting point (or depot), is whidely used in the transportation
sector. This problem is known as the Vehicle Routing Problem (VRP).

We define the TSP, given a complete graph $G = (V, E)$, as finding the tour (or Hamiltonian
circuit) with smallest total cost. For this paper we assume the cost $c_e$ of an edge $e = (v, w)$
is the euclidean distance between $v$ and $w$.
Given additionally a demand $d_v$ for each vertex $v$, a maximal capacity $C$ of goods per truck, a
number $M$ of available trucks and a special vertex $v_0$ that is the depot, we can define the VRP
as finding a set of at most $M$ tours with the least total cost. Each tour has to start and end at
the depot and has to have a total demand of at most $C$.
There are many variants with additional constraints or freedoms, but these are out of the scope of
this paper.

% Structure
The paper is organised as follows: in section~\ref{sec:heuristics} we will discuss the heuristics
used to generate the initial tours and routes. In section~\ref{sec:td} we will discuss how the
solutions are merged and how the treedecomposition is calculated and in section~\ref{sec:dp} we will
show the dynamic programming algorithms on the computed decompositions. In section~\ref{sec:results}
we will discuss the result and finally we will conclude in section~\ref{sec:conclusion}.



%
% Heuristics
%
\section{Heuristics}
\label{sec:heuristics}
Although many different heuristics have been tried to solve the Traveling Salesman Problem, there
are few that can compete with (variants of) the Lin Kernighan heuristic \cite{lin-kernighan}. We
will discuss the Lin Kernighan Helsgaun~\cite{lkh} variant in section~\ref{sec:lkh}.
For the Vehicle Routing Problem the currently best heuristics are tabu search algorithms.
Unfortunately they often require to finetune a lot of parameters and are focussed on specific
instances of VRP, rather than giving consistent solutions for all versions \cite{hmmz, where to find
a good citation for this one}. Other heuristics like the classic savings heuristic or the sweep
heuristic do give good solutions for all variants of VRP, but they can't get the results one
gets with the tabu heuristics.
% SWEEP????????

    \subsection{Lin Kernighan Helsgaun}
    \label{sec:lkh}
    Todo



%
% Merging and Tree Decomposition calculations
%
\section{Tree decomposition}
\label{sec:td}
% Merging
Once we have generated a couple of good tours for our original graph $G'=(V,E')$, we merge these
tours in one graph. All tours $E'_j \subset E'$, for $1 \leq j \leq} \#tours$ as found by the
heuristics are merged together into a new graph $G = (V, E)$, where $E=\bigcup E'_j}$). For this
graph $G$ we will compute a tree decomposition so that we can compute the optimal tour in this
reduced problem.

    % What is a tree decomposition
    \subsection{Tree decomposition and width}
    \label{sec:td-definition}
    A \emph{tree decomposition} of a graph $G=(V, E)$ is a tree $T=(W, F)$ and a \emph{bag}, or a set of
    vertices, $X_i \subset V$ for $i \in W$ such that:
    \begin{enumerate}
        \item $\bigcup_{i \in W} X_i = V,$
        \item for all $(u, v) \in E$ there is an $i \in W$ with $u, v \in X_i$ and
        \item for all $v \in V$, the set $W_v = \{i \in W: v \in X_i\}$ forms a connected subtree of $T$.
    \end{enumerate}
    The \emph{width} $k$ of the tree decomposition is $\max_{i \in W} |X_i| - 1$. The \emph{treewidth}
    of a graph $G$, is the minimum width among all tree decompositions of $G$.

    % TODO: example of graph with it's tree decomposition.

    In this paper we often work with the edge set corresponding to the bag $X_i$, $i \in W$, rather
    than the vertex set $X_i$ itself, and to that end we define $Y_i = \{(u,v) \in E: u,v \in X_i \}$.
    We say that a bag contains a vertex $v$ if $v \in X_i$ and that it contains an edge $e$ if $e \in Y_i$.

    % Todo: Minimum Degree Heuristic
    \subsection{Minimum Degree Heuristic}
    \label{sec:td-heuristic}
    Calculating the optimal treewidth or the optimal tree decompositions is an NP Hard problem
    \cite{find ref} so finding an optimal decomposition in reasonable time is infeasable unlesss
    $P=NP$. We also do not nescessarily need a tree decomposition of optimal width, we just need the
    width to be sufficiently small so that our DP algorithm runs fast enough.
    Therefore, we compute our tree decomposition with a heuristic: the minimum degree heuristic by
    Bodlaender and Koster~\cite{tw-upper-bounds}.

    Todo

    % TODO: trick of that other students paper to not merge all graphs?



%
% The dynamic programming algorithms
%
\section{Dynamic programming}
\label{sec:dp}
Provided the treewidth is small enough, the optimal solution on the merged graph and its tree
decomposition can be computed using dynamic programming algorithms. In the following sections we
explain their details.

    \subsection{Traveling Salesman}
    \label{sec:dp-tsp}
    % 2-connectedness: to make the graph disjoint, you need to remove at least 2 vertices
    % Simle graph: G is unweighted (?); undirected; no double edges; no self loops.
    Let $G=(V, E)$ be a simple graph with edge-weights $c_e$ and $T=(W, F)$ be the tree
    decomposition with width $k - 1$ and $X_i$ and $Y_i$ as defined in
    section~\ref{sec:td-definition}.
    Note that because $G$ is the result of a number of merged tours, it is 2-connected.
    The main idea of the algorithm is to find a series of disjoint paths and connect them together
    into a hamiltonian tour of minimum weight. Such a path is encoded using vertex degrees and a
    matching. Every vertex can have degree 0, 1 or 2. Vertices with degree 2 are already \emph{used}
    in a path, vertices with degree 1 are \emph{endpoints} of a path and vertices with degree 0 are
    \emph{free}, so not yet used in a path. For every pair of endpoints we have a \emph{matching}
    $\{u, v\}: u,v \in V$ to mark which vertices mark the endpoints of one path.

    We now define the function $F(X_i, degrees, matching)$ to be the minimum total cost of the
    edges in a series of paths starting and ending in bag $X_i$. The endpoints of these paths should
    be as in the matching, all vertices in bags below $X_i$ in the tree should be used (have a
    degree of 2) and the vertices in bag $X_i$ should have degrees as given in the $degrees$
    parameter.
    A vertex' degree can be satisfied in the bag itself by choosing an edge from $Y_i$ or by passing
    it on to one (or two) of the child bags. Formally this becomes
    \[
        F(X_i, D, M) = \min_{1 \leq p \leq P_i} (
            \sum_{j \in W :\text{ Parent}(j) = i} F(X_j, D_{p,j}, M_{p,j}) + \sum_{e \in E_p} c_e
        )
    \]
    for all $P_i$ ways of dividing $D$ and $M$ into the $D_{p,j}$ and $M_{p,j}$ sets and the
    corresponding $E_p \subset Y_i$.

    To find the different divisions we mainly focus on $D$, the set of degrees we have to 'spend'.
    We try all possible combinations of dividing the degrees of per vertex. For a given vertex $v$,
    assuming we have 2 degrees to spend, we first try to give it to each of the children. So for
    possibility $p$ and child bag $j$ we try to set the degree of $v$ in $D_{p,j}$ to 2.
    Afterwards, assuming we have at least 1 degree to spend for $v$, we try to use it as an endpoint
    for each of the vertices in all of the child bags. So the degrees of two vertices, $v$ and
    some other vertex $u$ in $D_{p,j}$ are set to 1. We try this for all (valid) combinations of $u$
    and $j$. At this step we also add the matching $\{u, v\}$ to $M_{p,j}$.
    Finally we also try not to assign $v$ to any of the child bags.
    Of course, vertices are only given to a child bag if it contains the vertex.

    For the remaining degrees that are not handled by any of the child bags we calculate a subset
    $E_p$ of $Y_i$. For non leaf bags these edges mainly glue paths from the children together in
    the paths as given in the $D$ and $M$ parameters. For the leaf bags there are of course no child
    bags to delegate the degrees to so it all has to be solved using the bags own edges.
    Note that the edge set is not allowed to introduce cycles, so in particular two endpoints in a
    matching are not allowed to be connected. The root bag is of course an exception, because there
    all paths are merged in a single cycle.
    If no valid edge set is found, $F(X_i, D, M) = \infty$.

    The overall algorithm then consists of a top down approach where we tabulate all entries for
    $F$, starting at the root and then recursively work downwards in the tree. Then the value of
    each table entry is finished bottem up as the recursion returns the values for children.

    % Overview:
    % - Idea: Disjoint paths
    % - Vertex degrees: 0, 1, 2
    % - Matchings for the 1's
    % - Talk about subproblems / partial tours / table-entries / functions
    % - The root bag
    % - Top down approach; recursive
    % - It tabulates the degrees+matchings

    \subsection{Vehicle Routing}
    \label{sec:dp-vrp}
    Todo

    \subsection{Speed}
    Although DP running time upperbounds of $O(n 3^k 2^{k^2}) - TODO???$ and $O(M n 3^k 2^{\dots})$
    are terrible, in practice these limits are never reached. This is because edges\dots TODO



%
% Results
%
\section{Results}
\label{sec:results}
Todo



%
% Conclusion
%
\section{Conclusion}
\label{sec:conclusion}
Todo



%
% Bibliography
%
\begin{thebibliography}{9}
    % The most important references
    \bibitem{cook-seymour}
        Cook, W., \& Seymour, P. (2003).
        \emph{Tour merging via branch-decomposition.}
        INFORMS Journal on Computing, 15(3), 233-248.

    \bibitem{lin-kernighan}
        Lin, S., \& Kernighan, B. W. (1973).
        \emph{An effective heuristic algorithm for the traveling-salesman problem.}
        Operations research, 21(2), 498-516.

    \bibitem{lkh}
        Helsgaun, K. (2009).
        \emph{General k-opt submoves for the Lin�Kernighan TSP heuristic.}
        Mathematical Programming Computation, 1(2-3), 119-163.

    \bibitem{tw-upper-bounds}
        Bodlaender, H. L., \& Koster, A. M. (2010).
        \emph{Treewidth computations I. Upper bounds.}
        Information and Computation, 208(3), 259-275.

\end{thebibliography}

\end{document}
